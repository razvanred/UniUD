\documentclass[11pt]{article}

\begin{document}
\title{Matematica in Latex}
\author{Calabrigo Massimo}
\date{\today}
\maketitle

Se voglio entrare in modalità matematica per scrivere delle formule, posso farlo scrivendo dollaro (x + 1) dollaro, e il risultato sarà: $ (x + 1) $\\
Per scrivere una formula matematica in modalità display (ovvero mettere la formula a capo su una linea solo per lei, 
senza il testo su quella linea), posso usare i doppi dollari: Questa formula calcola fibonacci $$ F(t) = F(t-1) + F(t-2) $$ in modalità ricorsiva.\\
In Latex, nella notazione matematica, si usano le parentesi graffe al posto di quelle rotonde quando si devono passare argomenti ad una funzione, oppure raggruppare una formula. 
Le tonde si usano invece per essere viste dall'utente, ma non vengono interpretate da Latex.
Ora vediamo le notazioni matematiche più comuni:\\
superscripts: Devo racchiudere tra parentesi graffe l'esponente di e, perchè altrimenti e prenderebbe come esponente solo il carattere immediatamente successivo 
e scriverebbe i successivi come numeri che moltiplicano e. Questi sono giusti:
$$ e^1 $$
$$ e^{18} $$
$$ e^{12+x^2} $$
Questo è sbagliato:
$$ e^16 $$
Subscripts: Devo racchiudere tra parentesi graffe la stringa dei caratteri a pedice, che sarà separata, dalla stringa che stà sopra, da un underscore. Questi sono giusti:
$$ x_1 $$
$$ x_{32} $$
$$ x_{76y+3} $$
$$ {x_y}_1 $$
Questo è sbagliato:
$$ x_12 $$
Lettere greche: si scrive un backslash, seguito dal nome della lettera greca.
Esempio. La formula dell'area del cerchio è: $$ \pi r^2 $$
Funzioni trigonometriche: si scrive backslash, seguito dal nome della funzione.
$$ y=\sin{x}+3 $$
Funzioni logaritmiche: si scrive backslash log.
$$ \log{x} $$
$$ \log_5{x} $$
$$ \ln{x} $$
Radici quadrate, cubiche, ...: si scrive backslash sqrt, e ha 2 parametri; il primo è il numero della radice (quadrato, cubo, ecc.), il secondo è la stringa sotto radice.\\
Ecco una radice quadrata: $ \sqrt{x+1} $\\
Ecco una radice ennesima: $ \sqrt[n]{{x+1}^2} $\\
Frazioni: si scrive backslash, seguito da frac{numeratore}{denominatore}\\
Questo bicchiere è pieno per $\frac{3}{4}$ della sua dimensione.\\
Se voglio che 3/4 sembri più grande, posso usare il comando displaystyle:\\
Questo bicchiere è pieno per $\displaystyle{\frac{3}{4}}$ della sua dimensione.\\

Parentesi:

Per scrivere le parentesi posso senplicente scriverle da tastiera senza comandi $(x[3y + 2])$. Se voglio 
scrivere il dollaro o le parentesi graffe devo mettere un backslash prima dei simboli: $ \{1,2,3\} $.\\
Se scrivo le parentesi tonde normalmente in una frazione mi esce questo risultato 
$$ 3(\frac{3}{2}) $$ ma possiamo migliorare questa scrittura in modo che le parentesi contengano interamente la frazione: $$ 3\left(\frac{3}{2}\right) $$
La stessa regola vale per tutti i simboli che racchiudono una frazione, per esempio se ho il valore assoluto devo fare $$ 3\left|\frac{2}{3}\right| $$
Tabelle: La tabella inizia con begin(tabular){} e finisce con end(tabular). begin prende come argomento n lettere c (column), per esempio 
se voglio una tabella con 5 colonne scriverei come input {ccccc}, separando le c con |, metto una linea di separazione tra quella colonna e la successiva.\\
Dentro al tabular devo inserire per ogni riga un numero di parametri pari al numero di colonne, e i parametri devono essere divisi 
da una \&. Ogni riga della tabella deve andare a capo, e se voglio mettere delle linee di separazione tra una riga e 
un altra devo scrivere backslash hline \\
\begin{tabular}{c|c|cc}
$x$ & 1 & 2 & 3 \\ \hline
$f(x)$ & 5 & 10 & 15 \\
\end{tabular}\\
Esempio di una tabella completa:\\
\begin{tabular}{|c|c|c|c|c|}
    \hline
    $x$ & 1 & 2 & 3 & 4 \\ \hline
    $f(x)$ & 10 & 20 & 30 & 40 \\ \hline
\end{tabular}\\
Serie di calcoli: Se ho bisogno di scrivere un'equazione con i relativi passaggi in ordine, posso usare una
eqnarray. L'equation array inizia con begin(eqnarray) e finisce con end(eqnarray).\\
\begin{eqnarray}
4x^2 + 2 = 3x^2 + 3 + x\\
x^2 = 1 + x\\
x^2 - x = 1\\
x(x-1) = 1\\
x\approx\pm1.72
\end{eqnarray}\\
Se vogliamo allineare le equazioni sull'uguale, dobbiamo aggiungere delle \& prima e dopo ogni uguale, e se 
vogliamo togliere i numeri a lato, al posto di eqnarray su begin e su end scriviamo eqnarray*:\\
\begin{eqnarray*}
    4x^2 + 2 &=& 3x^2 + 3 + x\\
    x^2 &=& 1 + x\\
    x^2 - x &=& 1\\
    x(x-1) &=& 1\\
    x&\approx&\pm1.72
\end{eqnarray*}
\end{document}