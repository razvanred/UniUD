\documentclass[11pt]{article}
\begin{document}
\title{Formattazione di Latex}
\author{Calabrigo Massimo}
\date{\today}
\maketitle

\tableofcontents

\section{Impostazione di un documento Latex}
\subsection{Dichiarazione del documento e Creazione del body}
N.B.: Tutti i comandi in Latex iniziano con un blackslash
il comando documentclass[]{} deve essere insertito all'inizio di ogni documento,
e come primo parametro ha la grandezza dei caratteri (11pt,12pt,...),
come secondo parametro ha il tipo di documento (article).

Sotto documentclass metto il body del testo, ovvero dove andrò a scrivere tutto il corpo del testo. Il body inizia da begin{document} e finisce a end{document}.\\
\subsection{Andare a capo}
Per andare a capo ci sono due modi:\\
Usare un soft return scrivendo 2 backslash
Usare un hard return lasciando una linea di testo vuota tra le 2 frasi che voglio separare.\\
\subsection{Scrivere il titolo}
Titolo e Headers: Possiamo scrivere titolo, autore e data; e avremmo bisogno di 4 righe di codice:\\
\begin{enumerate}
    \item title{titolo}(obbligatorio)
    \item author{nomeAutore}
    \item date
    \item maketitle(obbligatorio)
\end{enumerate}
\section{Sintassi di formattazione del testo}
\subsection{elenchi}
Elenchi puntati/numerati: ci sono 2 tipologie di elenchi entrambi racchiusi tra begin e end, e dopo l'end non 
bisogna mettere a capi:\\
Gli elenchi numerati (enumerate) hanno questa struttura:\\
\begin{enumerate}
\item albero di mele
    \begin{enumerate}
        \item mela1
        \item mela2
        \item mela3
        \begin{enumerate}
            \item seme1
            \item seme2
        \end{enumerate}
    \end{enumerate}
\item albero di pere
\item albero di banane
\end{enumerate}
E gli elenchi puntati (itemize), hanno questa struttura:\\
\begin{itemize}
    \item albero di mele
        \begin{itemize}
            \item mela1
            \item mela2
            \item mela3
            \begin{itemize}
                \item seme1
                \item seme2
            \end{itemize}
        \end{itemize}
    \item albero di pere
    \item albero di banane
\end{itemize}
Le due tipologie di elenchi puntati possono essere mescolate.\\
Posso anche scrivere delle stringhe al posto dei punti o dei numeri:\\
\begin{enumerate}
\item[Commutativa] $a+b=b+a$
\item[Associativa] $(a+b)+c=a+(b+c)$
\item[Distributiva] $a(b+c)=ab+ac$
\end{enumerate}
\subsection{Stile del testo}
Stringhe in grassetto/corsivo.
\begin{itemize}
\item grassetto: Questo \textbf{testo} è in grassetto.
\item corsivo: Questo \textit{testo} è in corsivo.
\item evidenziato: Questo \texttt{testo} è evidenziato.
\end{itemize}
\subsection{Formattazione del testo}
Posso posizionare del testo a sx, 
in centro o a dx dando come input a begin-end rispettivamente i comandi flushleft, 
center e flushright: 
\\
\begin{flushleft}
    Testo a sinistra
\end{flushleft}
\begin{center}
    testo in centro
\end{center}
\begin{flushright}
    testo a destra
\end{flushright}
\section{sezioni/capitoli}
Possiamo organizzare il testo tramite delle sezioni, come dei capitoli e con il comando backslash 
tableofcontents possiamo mostrare l'indice di tutte le section e subsection:\\
\section{Leggi di Newton}
    \subsection{Prima legge di Newton}
    Un oggetto che si muove di moto rettilineo uniforme in un sistema isolato, continuerà a muoversi di moto rettilineo uniforme.
    \subsection{Seconda legge di Newton}
    $F=mg$
    \subsection{Terza legge di Newton}
    Se un corpo A esercita una forza $F_{a,b}$ su di un corpo B, il corpo B eserciterà 
    una forza uguale e contraria $F_{b,a}$, sul corpo A.
\section{Cinematica}
    \subsection{Moto rettilineo uniforme}
    \subsection{Moto armonico}





\end{document}