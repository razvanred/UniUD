\documentclass[11pt]{article}

\usepackage{amsmath}

\begin{document}

\title{Materiale per il secondo compitino di gorni}
\author{Calabrigo Massimo}
\date{\today}
\maketitle

\tableofcontents

\section{Numeri non generabili da algoritmi}
Non tutti i numeri reali possono essere generati da algoritmi; Per dimostrare ciò ci serve l' \textbf{argomento diagonale}.\\
Rappresentiamo un elenco ordinato di tutti i numeri reali:
\begin{enumerate}
    \item $.............$
    \item $0.11111111...$
    \item $0.22222222...$
    \item $3.76535435...$
    \item $12.5454354...$
    \item $76.8479634...$
    \item $.............$
\end{enumerate}
e prendiamo una "foto" finita dell'insieme dei numeri reali infiniti:
\begin{enumerate}
    \item $0.11111$
    \item $0.22222$
    \item $3.76535$
    \item $12.54543$
    \item $76.84796$
\end{enumerate}
A questo punto prendiamo la n-esima cifra dopo la virgola alla n-esima riga, e diamogli un valore diverso da quella che ha:
\begin{enumerate}
    \item $0.|7|1111$
    \item $0.2|5|222$
    \item $3.76|8|35$
    \item $12.545|3|3$
    \item $76.8479|1|$
\end{enumerate}
Abbiamo appena creato un nuovo numero che non esisteva nell'elenco precedente.
\begin{enumerate}
    \item $0.11111$
    \item $0.22222$
    \item $0.75831$
    \item $3.76535$
    \item $12.54543$
    \item $76.84796$
\end{enumerate}
Ora aggiungiamo una cifra decimale ad ogni riga e otteniamo un nuovo gruppo finito, e da esso 
possiamo ricavare infiniti numeri che non facevano parte dell'elenco. La stessa cosa vale per l'insieme $R$ dei 
numeri reali, poichè abbiamo un elenco con $n = +\infty$, perciò anche la diagonale e i numeri 
da essa generati saranno infiniti, e non generabili da algoritmi.\\
\begin{enumerate}
    \item $0.111111$
    \item $0.222222$
    \item $0.758312$
    \item $3.765353$
    \item $12.545436$
    \item $76.847968$
\end{enumerate}
N.B.: la diagonale è solo un esempio per dimostrare che un algoritmo non può generare tutti i numeri reali, 
se al posto della diagonale, dicessimo di cambiare un valore casuale dopo la virgola per riga, ottenendo un nuovo numero, 
otterremmo lo stesso risultato di dimostrare l'impossibilità di generare tutti i numeri reali con gli algoritmi.
\section{Limiti con l'Hopital}
L'Hopital serve a gestire forme di limite indeterminate, altrimenti non risolvibili.\\
Possiamo risolvere le forme indeterminate: $\frac{\infty}{\infty}$ e $\frac{0}{0}$.\\
\begin{enumerate}
    \item se $lim_{x \rightarrow 0}f(x) = 0$ e $lim_{x \rightarrow 0}g(x) = 0$ e 
    $\frac{lim_{x \rightarrow 0}f^{'}(x)}{lim_{x \rightarrow 0}g^{'}(x)} = L$,\\ allora 
    $\frac{lim_{x \rightarrow 0}f(x)}{lim_{x \rightarrow 0}g(x)} = L$
    \item se $lim_{x \rightarrow \infty}f(x) = 0$ e $lim_{x \rightarrow \infty}g(x) = 0$ e 
    $\frac{lim_{x \rightarrow \infty}f^{'}(x)}{lim_{x \rightarrow \infty}g^{'}(x)} = L$,\\ allora 
    $\frac{lim_{x \rightarrow \infty}f(x)}{lim_{x \rightarrow \infty}g(x)} = L$
\end{enumerate}

\subsection{Procedimento per affrontare i limiti con l'hopital}
Quando affrontiamo i limiti con l'hopital, ci troveremo sempre davanti ad una situazione 
$f(x)/g(x)$, nel caso non fosse cosi basterà ricondurci a quella situazione facendo un 
denominatore comune.\\
Risolvere per prima $g(x)$, poichè potrebbe aiutare a capire quante volte io debba applicare poi 
l'hopital (per esempio se $g(x)$, si semplifica in $x^{3}$).\\
Bisogna guardare se ci sono situazioni in cui il limite può tendere da destra o da sinistra, 
e quindi fare i limiti $lim_{x->0^{+}}$ e $lim_{x->0^{-}}$.\\
Il procedimento sotto illustrato contiene tutti i passi per la risoluzione di un limite in questa 
forma, passi che devono essere eseguiti ricorsivamente, finchè non si arriva ad un risultato:\\
\begin{enumerate}
    \item Provare a sostituire il valore del limite;
    \item Provare a semplificare $f(x)$ con $g(x)$
    \item Provare a razionalizzare;
    \item Cercare possibili trasformazioni trigonometriche;
    \item Cercare possibili limiti notevoli;
    \item Fare la derivata;
\end{enumerate}

\section{Limiti e derivate dei seni e dei coseni}

\subsection{Derivate da ricordare}
\begin{center}
 \begin{tabular}{|c|c|c|}
 \hline
 Funzioni & Derivate:\\
 \hline \hline
 $arcSin(x)$ & $\frac{1}{\sqrt{1-x^{2}}}$\\
 \hline
 $arcCos(x)$ & $-\frac{1}{\sqrt{1-x^{2}}}$\\
 \hline
 $arcTan(x)$ & $\frac{1}{1+x^{2}}$\\
 \hline
 \end{tabular}
\end{center}
\subsection{Trasformazioni trigonometriche da ricordare}
\begin{itemize}
    \item $sin(2x) = 2*sin(x)*cos(x)$
    \item $cos(2x) = sin^{2}x + cos^{2}x$
    \item $tan(x) = \frac{sin(x)}{cos(x)}$
    \item $sin^{2}x + cos^{2}x = 1$
\end{itemize}
\section{Teorema dell'esisistenza degli zeri e funzioni continue}
\subsection{Funzioni continue}
\textbf{Definizione di continuità}:\\
$\forall \epsilon > 0$ e $\exists \delta > 0$ e $\forall x \in Dom(f)$
then $$|x - x_{0}| < \epsilon \rightarrow |f(x) - f(x_{0}| < \delta$$.\\
Quindi $f(x) = \frac{1}{x}$, è una funzione continua, perchè $x_{0} \notin Dom(f)$\\
\textbf{Proprietà delle funzioni continue:} Se f è una funzione continua, allora: $$lim_{x \rightarrow 0}(f(g(x))) = f(lim_{x \rightarrow 0}(g(x)))$$
Le combinazioni di funzioni continue danno origine a funzioni continue. \\\textbf{Esempio:} polinomi, esponenzioli, e funzioni trigonometriche 
sono funzioni continue, quindi $\frac{sin(x^{3} + e^{x+2})}{x^{2}+1}$ è  continua.\\
\subsection{Teorema dell'esisistenza degli zeri}
Sia $[a,b]$ un intervallo chiuso, e $f: [a,b] \rightarrow R$, e sia che $f(a)$ e $f(b)$, abbiano segni opposti, 
allora $\exists x_{s} \in [a,b]$ tale che $f(x_{s}) = 0$\\
Per vedere se esiste uno zero, uso un algoritmo iterativo.\\
Ci sono due casi di partenza
\begin{enumerate}
    \item $f(a_{0}) < 0$ e $f(b_{0}) > 0$
    \item $f(a_{0}) > 0$ e $f(b_{0}) < 0$
\end{enumerate}
Analizziamo il primo caso:\\
Troviamo un punto $c_{0} = \frac{a_{0} + b_{0}}{2}$. A questo punto vediamo i casi:
\[
  f(c_{i})=\begin{cases}
               < 0 \rightarrow cerchiamo c_{i+1} a destra, I_{i+1} = [c_{i+1},b]\\
               = 0 \rightarrow abbiamo finito\\
               > 0 \rightarrow cerchiamo c_{i+1} a sinistra, I_{i+1} = [a,c_{i+1}]
            \end{cases}
\]
Continuo a cercare $c_{i}$, dimezzando ogni volta l'intervallo e cercando a destra o a sinistra in base all'algoritmo 
sovrastante, finche non trovo $f(c_{i}) = 0$; a quel punto ho trovato che esiste uno zero.\\

\textbf{Per dimostrare il teorema mi manca una cosa però}, l'algoritmo potrebbe continuare all'infinito, 
devo dimostrare quando e perchè non è cosi.\\
Vado a vedere la lunghezza dell'intervallo $I_{i}$.Guardo alcuni casi:
\begin{itemize}
    \item $I_{0} = [a_{0},b_{0}]$ e quindi la lunghezza di $I_{0}$ è uguale a $\frac{b_{0} - a_{0}}{1}$
    \item $I_{1} = [a_{1},b_{1}]$ e quindi la lunghezza di $I_{1}$ è uguale a $\frac{b_{0} - a_{0}}{2}$
    \item $I_{2} = [a_{2},b_{2}]$ e quindi la lunghezza di $I_{2}$ è uguale a $\frac{b_{0} - a_{0}}{4}$
    \item $I_{3}$ = ...
    \item $I_{n} = [a_{n},b_{n}]$ e quindi la lunghezza di $I_{n}$ è uguale a $\frac{b_{0} - a_{0}}{2^{n}}$
\end{itemize}
Visto che $I_{n} = \frac{b_{0} - a_{0}}{2^{n}} = b_{n} - a_{n}$ possiamo riscriverlo come $b_{n} - a_{n} = \frac{b_{0} - a_{0}}{2^{n}}$\\
Noi sappiamo che $a_{i} <= a_{i+1}$ sarà sempre vero, e allo stesso modo $b_{i} >= b_{i+1}$. Quindi possiamo concludere che 
la successione $a_{n}$ è \textbf{debolmente crescente} (vuol dire che è non decrescente, ovvero che non cresce per forza sempre, ma può anche rimanere uguale)
, mentre $b_{n}$ è \textbf{debolmente decrescente}.\\
Sappiamo che le successione monotone hanno sempre un limite, quindi
\begin{itemize}
    \item $lim_{x \rightarrow \infty}(a_{n}) = a_{s}$
    \item $lim_{x \rightarrow \infty}(b_{n}) = b_{s}$
\end{itemize}
$$a_{0} <= a_{s} <= b_{s} <= b_{0}$$
Riassumiamo le Proprietà dell'intervallo:
\begin{enumerate}
    \item Visto che $I_{n} = \frac{b_{0} - a_{0}}{2^{n}}$, sappiamo che $I_{n}$ tende a zero.
    \item $f(a_{i})$ e $f(b_{i})$ hanno segni opposti
    \item $a_{n}$ è debolmente crescente
    \item $b_{n}$ è debolmente decrescente
    \item Esiste $lim_{x \rightarrow \infty}(a_{n}) = a_{s}$ ed è finito
    \item Esiste $lim_{x \rightarrow \infty}(b_{n}) = b_{s}$ ed è finito
\end{enumerate}
Facciamo il limite di $b_{n} - a_{n} = \frac{b_{0} - a_{0}}{2^{n}}$, per trovare il nostro $x_{s}$:
\begin{align*}
    &lim_{n \rightarrow \infty}(b_{n} - a_{n}) = lim_{n \rightarrow \infty}(\frac{b_{0} - a_{0}}{2^{n}})\\
    &b_{s} - a_{s} = 0\\
    &a_{s} = b_{s}
\end{align*}
Poniamo $x_{s} = a_{s} = b_{s}$\\
\textbf{Inizio parte facoltativa (serve per dimostrare che x tocca tutti gli intervalli $I_{i})$}\\
Dobbiamo dimostrare che $a_{s} <= x_{s} <= b_{s}$.\\
Io so che $a_{0} <= a_{1} <= ... <= a_{n} < b_{n} <= b_{n-1} <= ... <= b_{0}$.\\

Preso $m < n$, allora $a_{0} <= a_{1} <= ... <= a_{m} <= ... <= a_{n} < b_{n} <= b_{n-1} <= ... <= b_{m} <= ... <= b_{0}$, 
e quindi $$a_{m} < b_{n}$$
Preso $m > n$, allora $a_{0} <= a_{1} <= ... <= a_{n} <= ... <= a_{m} < b_{m} <= b_{m-1} <= ... <= b_{n} <= ... <= b_{0}$, 
e quindi $$a_{n} < b_{m}$$
Considero $a_{n} < b_{m}$ (o anche l'altra, a scelta). Ora posso dire che:
\begin{align*}
    &a_{n} < b_{m}\\
    &lim_{n \rightarrow \infty}(a_{n}) <= b_{m}\\
    &x_{s} <= b_{m}
\end{align*}
\begin{center}
    e
\end{center}
\begin{align*}
    &a_{n} < b_{m}\\
    &a_{n} <= lim_{m \rightarrow \infty}(b_{m})\\
    &a_{n} <= x_{s}
\end{align*}
\textbf{Fine parte facoltativa}\\
Ora che abbiamo dimostrato la formula possiamo usarla:
\begin{align*}
    f(a_{n}) < 0\\
    lim_{n \rightarrow \infty}(f(a_{n})) <= 0\\
    f(x_{s}) <= 0
\end{align*}
\begin{center}
    e
\end{center}
\begin{align*}
    f(b_{n}) > 0\\
    lim_{n \rightarrow \infty}(f(b_{n})) >= 0\\
    f(x_{s}) >= 0
\end{align*}
Quindi, grazie a $f(x_{s}) <= 0$ e $f(x_{s}) >= 0$, posso dire che 
$$f(x_{s}) = 0$$
\textbf{Quindi esiste uno zero della funzione}.
\begin{center}
   \textbf{FINE DIMOSTRAZIONE}
\end{center}
\section{Studio di funzione}
\subsection{Asintoti verticali, orizzontali e obliqui}
Quando dobbiamo cercare degli asintoti in una funzione, dobbiamo prima di tutto verificare se, e quali sono 
i buchi della funzione, ovvero quei punti dove la funzione non è definita, oltre che a $+\infty$ e $-\infty$, e bisogna cercare gli asintoti in quei punti.\\
Un \textbf{asintoto verticale} lo si trova quando $lim_{x \rightarrow x_{0}^{+,-}} = +- \infty$.\\
Per esempio, se $lim_{x \rightarrow 0^{+}} = \infty$ o $lim_{x \rightarrow 0^{+}} = -\infty$ o $lim_{x \rightarrow 0^{-}} = \infty$ o $lim_{x \rightarrow 0^{-}} = -\infty$, 
allora abbiamo trovato un asintoto verticale.\\
Un \textbf{asintoto orizzontale} si trova quando $lim_{x \rightarrow \infty} = l$.\\
Per esempio, se $lim_{x \rightarrow +\infty} = l$ oppure $lim_{x \rightarrow -\infty} = l$, allora abbiamo un asintoto orizzontale.\\
Per trovare i punti di intersezione tra l'asintoto orizzontale e la funzione posso mettere a sistema la funzione e l'equazione dell'asintoto:\\
$$
\begin{cases}
    y = f(x)\\
    y = l
\end{cases}
$$
Un \textbf{asintoto obliquio} è tale se, data la retta $y = mx + q$, allora vale il limite $lim_{x \rightarrow \infty}(f(x) - mx - q)$.\\
Quando bisogna vedere se c'è un asintoto obliquo bisogna procedere in questo modo:\\
\begin{enumerate}
    \item dobbiamo calcolare $y = mx + q$, quindi calcoliamo $m = lim_{x \rightarrow \infty}(\frac{f(x)}{x})$; se $m \in R$, allora continuo (calcolo q), se $m = 0$ ho un asintoto orizzontale (calcolo q), altrimenti non ho un asintoto obliquo.
    \item $q = lim_{x \rightarrow \infty}(f(x) - mx)$. Se $q \in R$ allora ho un asintoto obliquo.
\end{enumerate}
Esempio:$f(x) = \frac{x^{2}}{x-1}$, trova se esiste $y = mx + q$
\begin{align*}
    &m = lim_{x \rightarrow \infty}(f(x)/x)\\
    &m = lim_{x \rightarrow \infty}(\frac{x^{2}}{x-1}/x)\\
    &m = lim_{x \rightarrow \infty}(\frac{1}{(1-\frac{1}{x})}) = 1\\
    &q = lim_{x \rightarrow \infty}(f(x) - mx)\\
    &q = lim_{x \rightarrow \infty}(\frac{x^{2}}{x-1} - x)\\
    &q = lim_{x \rightarrow \infty}(\frac{x}{x-1})\\
    &q = lim_{x \rightarrow \infty}(\frac{1}{1 - \frac{1}{x}}) = 1\\
    &y = mx + q = x + 1
\end{align*}
\subsection{Come affrontare una funzione con logaritmi o arctan e gli zeri della funzione}
Affrontare questo tipo di funzioni non è poi tanto diverso dall'affrontare quelle polinomiali, ma con la differenza che bisogna stare più attenti sia 
nella parte di studio della funzione di partenza e nel determinare il dominio, che nella successiva parte delle derivate.\\
In questo tipo di esercizi, può essere chiesto di trovare gli zeri della funzione. Torna utile quindi il teorema di esistenza degli zeri.\\
Come lo applico? Una volta che ho fatto la derivata prima, e lo studio della crescenza/descrescenza, posso controllare ogni intervallo uno ad uno, 
e vedere se esiste uno zero. Ricorda che la funzione $f:[a,b] \rightarrow R$, deve essere continua nell'intervallo $[a,b]$. Per controllare posso usare 
le conoscenze che ho già acquisito nei passi precedenti, infatti andrò a vedere cosa vale la mia funzione nel punto a e cosa vale nel punto b, e da questo posso concludere 
se la funzione è passata per lo zero.\\
Per esempio se la mia tabella di crescenza/descrescenza è:\\
\begin{tabular}{|c|c|c|}
    \hline
    $0$ & $\sqrt{\frac{1}{2}}$ & $1$:\\
    \hline \hline
    $-$ & $-$ & $+$\\
    \hline
    $+$ & $+$ & $+$\\
    \hline\hline
    $-$ & $-$ & $+$\\
    \hline
\end{tabular}\\
e conosco che in 1 c'è un asintoto verticale, allora potrò dire che nell'intervallo $[\sqrt{\frac{1}{2}},1]$ c'è uno zero:
\begin{itemize}
    \item $[-\infty,0]$ no
    \item $[0,\sqrt{\frac{1}{2}}]$ no
    \item $[\sqrt{\frac{1}{2}},1]$ si
    \item $[1,\infty]$ no
\end{itemize}
\section{Primitive e integrali}
\subsection{Primitive e tabella}
L'operazione di antiderivazione è l'opposto della derivazione, $$\int(\frac{d}{dx}f(x)) = f(x)$$ Vediamo un paio di trasformazioni da $f(x)$ a $\int f(x)$:
\begin{center}
    \begin{tabular}{|c|c|}
        \hline
        $f(x)$ & $\int f(x)$\\
        \hline\hline
        $k$(costante) & $kx$\\
        \hline
        $x^{a}$ & $\frac{x^{a+1}}{a+1}$\\
        \hline
        $\frac{1}{x}$ & $ln(|x|)$\\
        \hline
        $sen(x)$ & $-cos(x)$\\
        \hline
        $cos(x)$ & $sin(x)$\\
        \hline
        $a^{x}$ & $\frac{a^{x}}{log(a)}$\\
        \hline
        $\frac{1}{1+x^{2}}$ & $arctg(x)$\\
        \hline
    \end{tabular}
\end{center}
Vediamo le proprietà delle operazioni tra integrali e le funzioni composte
\begin{center}
    \begin{itemize}
        \item $\int(f(x) \pm g(x))dx = \int(f(x))dx \pm \int(g(x))dx$
        \item $\int(k*f(x))dx = k*\int(f(x))dx$
    \end{itemize}
\end{center}
Una regola fondamentale è quella che riguarda le funzioni composte:
$$\int f^{'}(g(x)) * g^{'}(x) = f(g(x))$$
Esempio:
\begin{align*}
    \int(3*x^{2} * sin(x^{3}))\\
    g^{'}(x) * f^{'}(g(x))\\
    -cos(x^{3}) + c
\end{align*}
Da questa regola ne derivano varie:\\
\begin{tabular}{|c|c|}
    \hline
    Primitiva elementare & Primitiva composta\\
    \hline\hline
    $f(x) = \frac{1}{x}$ e $\int f(x)dx = ln(|x|) + c$ & $\int(\frac{g^{'}(x)}{g(x)}) = ln|g(x)| + c$\\
    \hline
    $f(x) = x^{n}$ e $\int f(x)dx = \frac{x^{n+1}}{n+1} + c$ & $\int(g^{'}(x)*g(x)^{n}) = \frac{g(x)^{n+1}}{n+1}  + c$\\
    \hline
    \dots & \dots\\
    \hline
\end{tabular}
\subsection{Risoluzione delle primitive}
Ci sono 4 tipi di primitive delle verifiche di gorni: 1 composta da polinomi, 2 composte da log(x), cos(x), sin(x), tan(x), arctan(x), ecc., e 1 per parti.\\
Vediamo come risolvere queste tipologie separatamente.\\
\textbf{Primitive composte da polinomi}\\
Quando dobbiamo affrontare questo tipo di primitive, non possiamo partire in quarta e fare le primitive; prima bisogna fare la \textbf{Divisione del polinomio}. 
Ora vedremo questo primo procedimento:\\
\begin{enumerate}
    \item Denominiamo $P(x)$ il polinomio numeratore e $D(x)$ il polinomio denominatore;
    \item Scriviamo $P(x) | D(x)$;
    \item Calcoliamo $gMax = \frac{termine di grado max di P(x)}{termine di grado massimo di D(x)}$;
    \item Scriviamo
    \begin{tabular}{c|c}
        $P(x)$ & $D(x)$\\
        \hline
               & $gMax$
    \end{tabular}
    \item Moltiplichiamo gMax per ogni termine di P(x), e scriviamo i risultati, cambiati di segno, nella riga sotto P(x):\\
    \begin{tabular}{c|c}
        $P(x)$ & $D(x)$\\
        \hline
        $D(x)*gMax*(-1)$ & $gMax$
    \end{tabular}
    \item Sommiamo la riga $P(x)$ con la riga $P(x)*gMax*(-1)$ (somma termine per termine), e riportiamo il risultato nella riga sottostante:\\
    \begin{tabular}{c|c}
        $P(x)$ & $D(x)$\\
        \hline
        $D(x)*gMax*(-1)$ & $gMax$\\
        \hline
        $P(x) + D(x)*gMax*(-1)$
    \end{tabular}
    \item Continuo finchè non arrivo ad avere un grado zero in $Q(x)$:\\
    \begin{tabular}{c|c}
        $P1(x)$ & $D(x)$\\
        \hline
        $D(x)1*gMax1*(-1)$ & $Q(x) = gMax1$, $gMax2, \dots, gMaxN$\\
        \hline
        $P1(x) + D(x)1*gMax1*(-1)$\\
        \hline
        $D(x)*gMax2*(-1)$\\
        \hline
        $P2(x) + D2(x)*gMax2*(-1)$\\
        \hline
        $\dots$\\
        \hline
        $PN(x) = \dots$
    \end{tabular}
    \item Ora posso scrivere il risultato, che sarà $Q(x) + \frac{PN(x)}{D(x)}$
\end{enumerate}
Facciamo un esempio: $f(x) = \frac{3x^3-2x^2+x}{6x^2+5x+1}$\\
$P(x) = 3x^3-2x^2+x$ e $D(x) = 6x^2+5x+1$.\\
\begin{enumerate}
    \item \begin{tabular}{c|c}
        $3x^3-2x^2+x$ & $6x^2+5x+1$\\
        \hline
    \end{tabular}\\
    \item \begin{tabular}{c|c}
        $3x^3-2x^2+x$ & $6x^2+5x+1$\\
        \hline
                    & $\frac{x}{2}$\\
        \hline
    \end{tabular}\\
    \item \begin{tabular}{c|c}
        $3x^3-2x^2+x$ & $6x^2+5x+1$\\
        \hline
        $-3x^3 -\frac{5x^{2}}{2} - \frac{x}{2} + 0$ & $\frac{x}{2}$\\
        \hline
    \end{tabular}\\
    \item \begin{tabular}{c|c}
        $3x^3-2x^2+x$ & $6x^2+5x+1$\\
        \hline
        $-3x^3 -\frac{5x^{2}}{2} - \frac{x}{2} + 0$ & $\frac{x}{2}$\\
        \hline
        $0x^3 -\frac{9x^{2}}{2} + \frac{x}{2} + 0$ & \\
        \hline
    \end{tabular}\\
    \item \begin{tabular}{c|c}
        $3x^3-2x^2+x$ & $6x^2+5x+1$\\
        \hline
        $-3x^3 -\frac{5x^{2}}{2} - \frac{x}{2} + 0$ & $\frac{x}{2}$, $-\frac{3}{4}$\\
        \hline
        $0x^3 -\frac{9x^{2}}{2} + \frac{x}{2} + 0$ & \\
        \hline
    \end{tabular}\\
    \item \begin{tabular}{c|c}
        $3x^3-2x^2+x$ & $6x^2+5x+1$\\
        \hline
        $-3x^3 -\frac{5x^{2}}{2} - \frac{x}{2} + 0$ & $\frac{x}{2}$, $-\frac{3}{4}$\\
        \hline
        $0x^3 -\frac{9x^{2}}{2} + \frac{x}{2} + 0$ & \\
        \hline
        $0x^3 -\frac{18x^{2}}{4} + \frac{15x}{4} + 0$ & \\
    \end{tabular}\\
    \item \begin{tabular}{c|c}
        $3x^3-2x^2+x$ & $6x^2+5x+1$\\
        \hline
        $-3x^3 -\frac{5x^{2}}{2} - \frac{x}{2} + 0$ & $\frac{x}{2}$, $-\frac{3}{4}$\\
        \hline
        $0x^3 -\frac{9x^{2}}{2} + \frac{x}{2} + 0$ & \\
        \hline
        $0x^3 -\frac{18x^{2}}{4} + \frac{15x}{4} + 0$ & \\
        \hline
        $0x^3 + 0x^{2} + \frac{17x}{4} + \frac{3}{4}$ & \\
    \end{tabular}\\
    \item $PN(x) = \frac{17x}{4} + \frac{3}{4}$ e $Q(x) = \frac{x}{2} - \frac{3}{4}$.\\
    Quindi possiamo riscrivere
    \begin{align*}
        &f(x) = \frac{3x^3-2x^2+x}{6x^2+5x+1}\\
        &f(x) = Q(x) + \frac{PN(x)}{D(x)}\\
        &f(x) = \frac{x}{2} - \frac{3}{4} + \frac{\frac{17x}{4} + \frac{3}{4}}{6x^2+5x+1}
    \end{align*} 
\end{enumerate}
\end{document}
