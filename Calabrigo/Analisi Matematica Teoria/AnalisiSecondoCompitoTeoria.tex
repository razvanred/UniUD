\documentclass[11pt]{article}

\usepackage{amsmath}

\begin{document}

\title{Materiale per il secondo compitino di gorni}
\author{Calabrigo Massimo}
\date{\today}
\maketitle

\tableofcontents

\section{Numeri non generabili da algoritmi}
Non tutti i numeri reali possono essere generati da algoritmi; Per dimostrare ciò ci serve l' \textbf{argomento diagonale}.\\
Rappresentiamo un elenco ordinato di tutti i numeri reali:
\begin{enumerate}
    \item $.............$
    \item $0.11111111...$
    \item $0.22222222...$
    \item $3.76535435...$
    \item $12.5454354...$
    \item $76.8479634...$
    \item $.............$
\end{enumerate}
e prendiamo una "foto" finita dell'insieme dei numeri reali infiniti:
\begin{enumerate}
    \item $0.11111$
    \item $0.22222$
    \item $3.76535$
    \item $12.54543$
    \item $76.84796$
\end{enumerate}
A questo punto prendiamo la n-esima cifra dopo la virgola alla n-esima riga, e diamogli un valore diverso da quella che ha:
\begin{enumerate}
    \item $0.|7|1111$
    \item $0.2|5|222$
    \item $3.76|8|35$
    \item $12.545|3|3$
    \item $76.8479|1|$
\end{enumerate}
Abbiamo appena creato un nuovo numero che non esisteva nell'elenco precedente.
\begin{enumerate}
    \item $0.11111$
    \item $0.22222$
    \item $0.75831$
    \item $3.76535$
    \item $12.54543$
    \item $76.84796$
\end{enumerate}
Ora aggiungiamo una cifra decimale ad ogni riga e otteniamo un nuovo gruppo finito, e da esso 
possiamo ricavare infiniti numeri che non facevano parte dell'elenco. La stessa cosa vale per l'insieme $R$ dei 
numeri reali, poichè abbiamo un elenco con $n = +\infty$, perciò anche la diagonale e i numeri 
da essa generati saranno infiniti, e non generabili da algoritmi.\\
\begin{enumerate}
    \item $0.111111$
    \item $0.222222$
    \item $0.758312$
    \item $3.765353$
    \item $12.545436$
    \item $76.847968$
\end{enumerate}
N.B.: la diagonale è solo un esempio per dimostrare che un algoritmo non può generare tutti i numeri reali, 
se al posto della diagonale, dicessimo di cambiare un valore casuale dopo la virgola per riga, ottenendo un nuovo numero, 
otterremmo lo stesso risultato di dimostrare l'impossibilità di generare tutti i numeri reali con gli algoritmi.
\section{Limiti con l'Hopital}
L'Hopital serve a gestire forme di limite indeterminate, altrimenti non risolvibili.\\
Possiamo risolvere le forme indeterminate: $\frac{\infty}{\infty}$ e $\frac{0}{0}$.\\
\begin{enumerate}
    \item se $lim_{x \rightarrow 0}f(x) = 0$ e $lim_{x \rightarrow 0}g(x) = 0$ e 
    $\frac{lim_{x \rightarrow 0}f^{'}(x)}{lim_{x \rightarrow 0}g^{'}(x)} = L$,\\ allora 
    $\frac{lim_{x \rightarrow 0}f(x)}{lim_{x \rightarrow 0}g(x)} = L$
    \item se $lim_{x \rightarrow \infty}f(x) = 0$ e $lim_{x \rightarrow \infty}g(x) = 0$ e 
    $\frac{lim_{x \rightarrow \infty}f^{'}(x)}{lim_{x \rightarrow \infty}g^{'}(x)} = L$,\\ allora 
    $\frac{lim_{x \rightarrow \infty}f(x)}{lim_{x \rightarrow \infty}g(x)} = L$
\end{enumerate}

\subsection{Procedimento per affrontare i limiti con l'hopital}
Quando affrontiamo i limiti con l'hopital, ci troveremo sempre davanti ad una situazione 
$f(x)/g(x)$, nel caso non fosse cosi basterà ricondurci a quella situazione facendo un 
denominatore comune.\\
Risolvere per prima $g(x)$, poichè potrebbe aiutare a capire quante volte io debba applicare poi 
l'hopital (per esempio se $g(x)$, si semplifica in $x^{3}$).\\
Bisogna guardare se ci sono situazioni in cui il limite può tendere da destra o da sinistra, 
e quindi fare i limiti $lim_{x->0^{+}}$ e $lim_{x->0^{-}}$.\\
Il procedimento sotto illustrato contiene tutti i passi per la risoluzione di un limite in questa 
forma, passi che devono essere eseguiti ricorsivamente, finchè non si arriva ad un risultato:\\
\begin{enumerate}
    \item Provare a sostituire il valore del limite;
    \item Provare a semplificare $f(x)$ con $g(x)$
    \item Provare a razionalizzare;
    \item Cercare possibili trasformazioni trigonometriche;
    \item Cercare possibili limiti notevoli;
    \item Fare la derivata;
\end{enumerate}

\section{Limiti e derivate dei seni e dei coseni}

\subsection{Derivate da ricordare}
\begin{center}
 \begin{tabular}{|c|c|c|}
 \hline
 Funzioni & Derivate:\\
 \hline \hline
 $arcSin(x)$ & $\frac{1}{\sqrt{1-x^{2}}}$\\
 \hline
 $arcCos(x)$ & $-\frac{1}{\sqrt{1-x^{2}}}$\\
 \hline
 $arcTan(x)$ & $\frac{1}{1+x^{2}}$\\
 \hline
 \end{tabular}
\end{center}
\subsection{Trasformazioni trigonometriche da ricordare}
\begin{itemize}
    \item $sin(2x) = 2*sin(x)*cos(x)$
    \item $cos(2x) = sin^{2}x + cos^{2}x$
    \item $tan(x) = \frac{sin(x)}{cos(x)}$
    \item $sin^{2}x + cos^{2}x = 1$
\end{itemize}
\section{Teorema dell'esisistenza degli zeri e funzioni continue}
\subsection{Funzioni continue}
\textbf{Definizione di continuità}:\\
$\forall \epsilon > 0$ e $\exists \delta > 0$ e $\forall x \in Dom(f)$
then $$|x - x_{0}| < \epsilon \rightarrow |f(x) - f(x_{0}| < \delta$$.\\
Quindi $f(x) = \frac{1}{x}$, è una funzione continua, perchè $x_{0} \notin Dom(f)$\\
\textbf{Proprietà delle funzioni continue:} Se f è una funzione continua, allora: $$lim_{x \rightarrow 0}(f(g(x))) = f(lim_{x \rightarrow 0}(g(x)))$$
Le combinazioni di funzioni continue danno origine a funzioni continue. \\\textbf{Esempio:} polinomi, esponenzioli, e funzioni trigonometriche 
sono funzioni continue, quindi $\frac{sin(x^{3} + e^{x+2})}{x^{2}+1}$ è  continua.\\
\subsection{Teorema dell'esisistenza degli zeri}
Sia $[a,b]$ un intervallo chiuso, e $f: [a,b] \rightarrow R$, e sia che $f(a)$ e $f(b)$, abbiano segni opposti, 
allora $\exists x_{s} \in [a,b]$ tale che $f(x_{s}) = 0$\\
Per vedere se esiste uno zero, uso un algoritmo iterativo.\\
Ci sono due casi di partenza
\begin{enumerate}
    \item $f(a_{0}) < 0$ e $f(b_{0}) > 0$
    \item $f(a_{0}) > 0$ e $f(b_{0}) < 0$
\end{enumerate}
Analizziamo il primo caso:\\
Troviamo un punto $c_{0} = \frac{a_{0} + b_{0}}{2}$. A questo punto vediamo i casi:
\[
  f(c_{i})=\begin{cases}
               < 0 \rightarrow cerchiamo c_{i+1} a destra, I_{i+1} = [c_{i+1},b]\\
               = 0 \rightarrow abbiamo finito\\
               > 0 \rightarrow cerchiamo c_{i+1} a sinistra, I_{i+1} = [a,c_{i+1}]
            \end{cases}
\]
Continuo a cercare $c_{i}$, dimezzando ogni volta l'intervallo e cercando a destra o a sinistra in base all'algoritmo 
sovrastante, finche non trovo $f(c_{i}) = 0$; a quel punto ho trovato che esiste uno zero.\\

\textbf{Per dimostrare il teorema mi manca una cosa però}, l'algoritmo potrebbe continuare all'infinito, 
devo dimostrare quando e perchè non è cosi.\\
Vado a vedere la lunghezza dell'intervallo $I_{i}$.Guardo alcuni casi:
\begin{itemize}
    \item $I_{0} = [a_{0},b_{0}]$ e quindi la lunghezza di $I_{0}$ è uguale a $\frac{b_{0} - a_{0}}{1}$
    \item $I_{1} = [a_{1},b_{1}]$ e quindi la lunghezza di $I_{1}$ è uguale a $\frac{b_{0} - a_{0}}{2}$
    \item $I_{2} = [a_{2},b_{2}]$ e quindi la lunghezza di $I_{2}$ è uguale a $\frac{b_{0} - a_{0}}{4}$
    \item $I_{3}$ = ...
    \item $I_{n} = [a_{n},b_{n}]$ e quindi la lunghezza di $I_{n}$ è uguale a $\frac{b_{0} - a_{0}}{2^{n}}$
\end{itemize}
Visto che $I_{n} = \frac{b_{0} - a_{0}}{2^{n}} = b_{n} - a_{n}$ possiamo riscriverlo come $b_{n} - a_{n} = \frac{b_{0} - a_{0}}{2^{n}}$\\
Noi sappiamo che $a_{i} <= a_{i+1}$ sarà sempre vero, e allo stesso modo $b_{i} >= b_{i+1}$. Quindi possiamo concludere che 
la successione $a_{n}$ è \textbf{debolmente crescente} (vuol dire che è non decrescente, ovvero che non cresce per forza sempre, ma può anche rimanere uguale)
, mentre $b_{n}$ è \textbf{debolmente decrescente}.\\
Sappiamo che le successione monotone hanno sempre un limite, quindi
\begin{itemize}
    \item $lim_{x \rightarrow \infty}(a_{n}) = a_{s}$
    \item $lim_{x \rightarrow \infty}(b_{n}) = b_{s}$
\end{itemize}
$$a_{0} <= a_{s} <= b_{s} <= b_{0}$$
Riassumiamo le Proprietà dell'intervallo:
\begin{enumerate}
    \item Visto che $I_{n} = \frac{b_{0} - a_{0}}{2^{n}}$, sappiamo che $I_{n}$ tende a zero.
    \item $f(a_{i})$ e $f(b_{i})$ hanno segni opposti
    \item $a_{n}$ è debolmente crescente
    \item $b_{n}$ è debolmente decrescente
    \item Esiste $lim_{x \rightarrow \infty}(a_{n}) = a_{s}$ ed è finito
    \item Esiste $lim_{x \rightarrow \infty}(b_{n}) = b_{s}$ ed è finito
\end{enumerate}
Facciamo il limite di $b_{n} - a_{n} = \frac{b_{0} - a_{0}}{2^{n}}$, per trovare il nostro $x_{s}$:
\begin{align}
    &lim_{n \rightarrow \infty}(b_{n} - a_{n}) = lim_{n \rightarrow \infty}(\frac{b_{0} - a_{0}}{2^{n}})\\
    &b_{s} - a_{s} = 0\\
    &a_{s} = b_{s}
\end{align}
Poniamo $x_{s} = a_{s} = b_{s}$\\
\textbf{Inizio parte facoltativa (serve per dimostrare che x tocca tutti gli intervalli $I_{i})$}\\
Dobbiamo dimostrare che $a_{s} <= x_{s} <= b_{s}$.\\
Io so che $a_{0} <= a_{1} <= ... <= a_{n} < b_{n} <= b_{n-1} <= ... <= b_{0}$.\\

Preso $m < n$, allora $a_{0} <= a_{1} <= ... <= a_{m} <= ... <= a_{n} < b_{n} <= b_{n-1} <= ... <= b_{m} <= ... <= b_{0}$, 
e quindi $$a_{m} < b_{n}$$
Preso $m > n$, allora $a_{0} <= a_{1} <= ... <= a_{n} <= ... <= a_{m} < b_{m} <= b_{m-1} <= ... <= b_{n} <= ... <= b_{0}$, 
e quindi $$a_{n} < b_{m}$$
Considero $a_{n} < b_{m}$ (o anche l'altra, a scelta). Ora posso dire che:
\begin{align}
    &a_{n} < b_{m}\\
    &lim_{n \rightarrow \infty}(a_{n}) <= b_{m}\\
    &x_{s} <= b_{m}
\end{align}
\begin{center}
    e
\end{center}
\begin{align}
    &a_{n} < b_{m}\\
    &a_{n} <= lim_{m \rightarrow \infty}(b_{m})\\
    &a_{n} <= x_{s}
\end{align}
\textbf{Fine parte facoltativa}\\
Ora che abbiamo dimostrato la formula possiamo usarla:
\begin{align}
    f(a_{n}) < 0\\
    lim_{n \rightarrow \infty}(f(a_{n})) <= 0\\
    f(x_{s}) <= 0
\end{align}
\begin{center}
    e
\end{center}
\begin{align}
    f(b_{n}) > 0\\
    lim_{n \rightarrow \infty}(f(b_{n})) >= 0\\
    f(x_{s}) >= 0
\end{align}
Quindi, grazie a $f(x_{s}) <= 0$ e $f(x_{s}) >= 0$, posso dire che 
$$f(x_{s}) = 0$$
\textbf{Quindi esiste uno zero della funzione}.
\begin{center}
   \textbf{FINE DIMOSTRAZIONE}
\end{center}
\end{document}