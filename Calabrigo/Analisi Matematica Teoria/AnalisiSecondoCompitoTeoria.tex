\documentclass[11pt]{article}
\DeclareUnicodeCharacter{030A}{}

\usepackage{amsmath}

\begin{document}

\title{Materiale per il secondo compitino di gorni}
\author{Calabrigo Massimo}
\date{\today}
\maketitle

\tableofcontents

\section{Numeri non generabili da algoritmi}
Non tutti i numeri reali possono essere generati da algoritmi; Per dimostrare ciò ci serve l' \textbf{argomento diagonale}.\\
Rappresentiamo un elenco ordinato di tutti i numeri reali:
\begin{enumerate}
    \item $.............$
    \item $0.11111111...$
    \item $0.22222222...$
    \item $3.76535435...$
    \item $12.5454354...$
    \item $76.8479634...$
    \item $.............$
\end{enumerate}
e prendiamo una "foto" finita dell'insieme dei numeri reali infiniti:
\begin{enumerate}
    \item $0.11111$
    \item $0.22222$
    \item $3.76535$
    \item $12.54543$
    \item $76.84796$
\end{enumerate}
A questo punto prendiamo la n-esima cifra dopo la virgola alla n-esima riga, e diamogli un valore diverso da quella che ha:
\begin{enumerate}
    \item $0.|7|1111$
    \item $0.2|5|222$
    \item $3.76|8|35$
    \item $12.545|3|3$
    \item $76.8479|1|$
\end{enumerate}
Abbiamo appena creato un nuovo numero che non esisteva nell'elenco precedente.
\begin{enumerate}
    \item $0.11111$
    \item $0.22222$
    \item $0.75831$
    \item $3.76535$
    \item $12.54543$
    \item $76.84796$
\end{enumerate}
Ora aggiungiamo una cifra decimale ad ogni riga e otteniamo un nuovo gruppo finito, e da esso 
possiamo ricavare infiniti numeri che non facevano parte dell'elenco. La stessa cosa vale per l'insieme $R$ dei 
numeri reali, poichè abbiamo un elenco con $n = +\infty$, perciò anche la diagonale e i numeri 
da essa generati saranno infiniti, e non generabili da algoritmi.\\
\begin{enumerate}
    \item $0.111111$
    \item $0.222222$
    \item $0.758312$
    \item $3.765353$
    \item $12.545436$
    \item $76.847968$
\end{enumerate}
N.B.: la diagonale è solo un esempio per dimostrare che un algoritmo non può generare tutti i numeri reali, 
se al posto della diagonale, dicessimo di cambiare un valore casuale dopo la virgola per riga, ottenendo un nuovo numero, 
otterremmo lo stesso risultato di dimostrare l'impossibilità di generare tutti i numeri reali con gli algoritmi.
\section{Disuguaglianza di Bernoulli}
La disuguaglianza di Bernoulli è una formula che aiuta a risolvere certe dimostrazioni per induzioni che usano i simboli $\ge$ e $\le$.\\
La disuguaglianza di Bermoulli dice che: $$(1+x)^{n} \ge (1 + nx) ; \forall n \in N, \forall x \ge 0$$ Dimostriamolo per induzione:\\
\begin{center}
    $P(n) = ((1+x)^{n} \ge (1+nx)) ; \forall n \in N, \forall x \ge 0$\\
    Caso Base:\\
    $P(0) = 1 \ge 1$\\
    Tesi induttiva:\\
    $P(n+1) = ((1+x)^{n+1} \ge (1 + (n+1)x))$\\
    Passo induttivo:\\
    \begin{align*}
        &(1+x)^{n}*(1+x) \ge (1 + (n+1)x)\\
        &(1+nx)*(1+x) \ge (1 + (n+1)x)\\
        &1+x+nx+nx^{2} \ge 1 + nx + x\\
        &nx^{2} \ge 0\\
    \end{align*}
    $nx^{2} \ge 0$ è sempre vero $\forall n \in N, \forall x \in R$, quindi $(1+nx)*(1+x) \ge (1 + (n+1)x)$ è vero, e per 
    la proprietà transitiva, detti $A = ((1+x)^{n}*(1+x))$, $B = ((1+nx)*(1+x))$ e $C = ((1 + (n+1)x))$ se $A \ge B \ge C$, allora $A \ge C$
     per la proprietà transitiva.\\
     \textbf{FINE DIMOSTRAZIONE}
\end{center}
\section{Teorema del limite della somma di 2 funzioni aventi limite finito}
Il teorema dice che $$(lim_{x \rightarrow x_{0}}f(x) = m) + (lim_{x \rightarrow x_{0}}g(x) = n) = (lim_{x \rightarrow x_{0}}(f(x)+g(x)) = m+n)$$
Dobbiamo dimostrarlo:\\
Possiamo riscriverlo con i moduli $lim_{x \rightarrow x_{0}}f(x) = m$ è come scrivere $|f(x)-m| < \epsilon_{1}$ e $lim_{x \rightarrow x_{0}}g(x) = n$ è come scrivere $|f(x)-n| < \epsilon_{2}$.\\
Quindi possiamo riscrivere $(lim_{x \rightarrow x_{0}}(f(x)+g(x)) = m+n)$ come $|f(x)+g(x)-m-n| < \epsilon_{1} + \epsilon_{2}$, che sarà la nostra ipotesi.\\
Dimostrazione:\\
\begin{align*}
    &lim_{x \rightarrow x_{0}}f(x) + lim_{x \rightarrow x_{0}}g(x) = m + n\\
    &|f(x)-m|<\epsilon_{1} , |g(x)-n|<\epsilon_{2}\\
    &|f(x)+g(x)-m-n| < |f(x)-m| + |g(x)-n| < \epsilon_{1} + \epsilon_{2}\\
    &|f(x)+g(x)-m-n| < \epsilon_{1} + \epsilon_{2}
\end{align*}
Quindi abbiamo dimostrato l'ipotesi.
\begin{center}
    \textbf{FINE DIMOSTRAZIONE}
\end{center}
\section{La successione fondamentale $(1+\frac{1}{x})^{x}$}
\subsection{crescenza della successione fondamentale}
La successione fondamentale $(1+\frac{1}{x})^{x}$ è crescente, dimostriamolo:\\
Detto $a_{n} = (1+\frac{1}{x})^{x}$, scriviamo $a_{n} \le a_{n+1}$.\\
Per motivi di calcoli, non dimostriamo la formula sovrastante, ma dimostriamo $a_{n-1} \le a_{n}$ e 
riscriviamola come $\frac{a_{n}}{a_{n-1}} \ge 1$ e dimostriamola.\\
\begin{align*}
    &\frac{(1+\frac{1}{n})^{n}}{(1+\frac{1}{n-1})^{n-1}}\\
    &\frac{(\frac{n+1}{n})^{n}}{(\frac{n}{n-1})^{n}} * \frac{n}{n-1}\\
    &(\frac{n+1}{n}*\frac{n-1}{n})^{n}*\frac{n}{n-1}\\
    &(\frac{n^{2}-1}{n^{2}})^{n}* \frac{n}{n-1}\\
    &(1- \frac{1}{n^{2}})^{n}*\frac{n}{n+1}
\end{align*}
Ora uso Bernoulli $(x+1)^{n} \ge 1 +nx$, pongo $x = -\frac{1}{n^{2}}$:\\
Per applicare Bernoulli devo rispettare le condizioni: $n \in N, x \ge 0$, quindi nel nostro caso abbiamo come 
ipotesi iniziale $n \ge 2$, quindi sappiamo che $n \in N$, ora verifichiamo che $x \ge 0$, ovvero $-\frac{1}{n^{2}} \ge 0$.\\
\begin{align*}
    &-\frac{1}{n^{2}} \ge 0\\
    &\frac{1}{n^{2}} \le 0\\
    &n^{̊2} \ge 0 \forall n \in R
\end{align*}
Abbiamo verificato di rispettare le condizioni di Bernoulli, quindi possiamo scrivere: $1 - \frac{1}{x^{2}} \ge 1 - \frac{1}{n^{2}}*n = 1 - \frac{1}{n}$\\
$(1 - \frac{1}{n})*(\frac{n}{n-1}) = \frac{n-1}{n}*\frac{n}{n-1} = 1$\\
Quindi abbiamo ottenuto che $\frac{a_{n}}{a_{n-1}} \ge 1$ è uguale a $1 \ge 1$, ed è vero.\\
Quindi abbiamo dimostrato il teorema\\
\begin{center}
    \textbf{FINE DIMOSTRAZIONE}
\end{center}
\section{Limiti con l'Hopital}
L'Hopital serve a gestire forme di limite indeterminate, altrimenti non risolvibili.\\
Possiamo risolvere le forme indeterminate: $\frac{\infty}{\infty}$ e $\frac{0}{0}$.\\
Se esiste un intervallo $I \rightarrow R$, due funzioni f e g continue in I e derivabili nei punti interni di $I$, e $g^{'}(x) \neq 0$. Allora
\begin{enumerate}
    \item se $lim_{x \rightarrow x_{0}}f(x) = 0$ e $lim_{x \rightarrow x_{0}}g(x) = 0$ e 
    $\frac{lim_{x \rightarrow x_{0}}f^{'}(x)}{lim_{x \rightarrow x_{0}}g^{'}(x)} = L$,\\ allora 
    $\frac{lim_{x \rightarrow x_{0}}f(x)}{lim_{x \rightarrow x_{0}}g(x)} = L$
    \item se $lim_{x \rightarrow x_{0}}f(x) = \infty$ e $lim_{x \rightarrow x_{0}}g(x) = \infty$ e 
    $\frac{lim_{x \rightarrow x_{0}}f^{'}(x)}{lim_{x \rightarrow x_{0}}g^{'}(x)} = L$,\\ allora 
    $\frac{lim_{x \rightarrow x_{0}}f(x)}{lim_{x \rightarrow x_{0}}g(x)} = L$
\end{enumerate}
Dimostrazione:\\
Preso un intervallo I di questo tipo $[x_{0},\infty)$, per ipotesi $lim_{x \rightarrow x_{0}}f(x) = lim_{x \rightarrow x_{0}}g(x) = 0$, vediamo 2 casi.\\
Se $x = x_{0}$, e x è estremo inferiore di I, allora $lim_{x \rightarrow x_{0}}f(x) = 0$, e visto che la funzione è continua nell'intervallo $lim_{x \rightarrow x_{0}}f(x) = f(x_{0}) = 0$, 
lo stesso vale per $lim_{x \rightarrow x_{0}}g(x) = g(x_{0}) = 0$\\
Se $x > x_{0}$, allora applico cauchy e dico che $\frac{f(x)}{g(x)} = \frac{f(x)-f(x_{0})}{g(x)-g(x_{0})} = \frac{f^{'}c(x)}{g^{'}}c(x)$.\\
Ora vediamo che $x_{0} < c(x) < x$, e per x che tende a 0, allora anche $c(x)$ tende a 0.\\
Quindi possiamo scrivere $lim_{x \rightarrow x_{0}}\frac{f^{'}c(x)}{g^{'}c(x)}$ con $t = c(x)$ e $c(x)$ che tende a zero, allora $lim_{t \rightarrow x_{0}}\frac{f^{'}(t)}{g^{'}(t)} = L$ per ipotesi, 
quindi anche $lim_{x \rightarrow x_{0}}\frac{f(x)}{g(x)} = L$ e abbiamo dimostrato la tesi.\\
\begin{center}
    \textbf{FINE DIMOSTRAZIONE}
\end{center}
\subsection{Procedimento per affrontare i limiti con l'hopital}
Quando affrontiamo i limiti con l'hopital, ci troveremo sempre davanti ad una situazione 
$f(x)/g(x)$, nel caso non fosse cosi basterà ricondurci a quella situazione facendo un 
denominatore comune.\\
Risolvere per prima $g(x)$, poichè potrebbe aiutare a capire quante volte io debba applicare poi 
l'hopital (per esempio se $g(x)$, si semplifica in $x^{3}$).\\
Bisogna guardare se ci sono situazioni in cui il limite può tendere da destra o da sinistra, 
e quindi fare i limiti $lim_{x->0^{+}}$ e $lim_{x->0^{-}}$.\\
Il procedimento sotto illustrato contiene tutti i passi per la risoluzione di un limite in questa 
forma, passi che devono essere eseguiti ricorsivamente, finchè non si arriva ad un risultato:\\
\begin{enumerate}
    \item Provare a sostituire il valore del limite;
    \item Provare a semplificare $f(x)$ con $g(x)$
    \item Provare a razionalizzare;
    \item Cercare possibili trasformazioni trigonometriche;
    \item Cercare possibili limiti notevoli;
    \item Fare la derivata;
\end{enumerate}
\section{Limiti e derivate dei seni e dei coseni}
\subsection{Definizione Derivata}
Una derivata è definita come il limite del rapporto incremenetale, e si scrive $$lim_{h \rightarrow 0}\frac{f(x+h)-f(x)}{h}$$ oppure 
$$lim_{x \rightarrow x_{0}}\frac{f(x)-f(x_{0})}{x-x_{0}}$$
Una funzione derivata è sempre continua, ma una funzione continua non è sempre derivabile; come esempio di funzione continua non derivabile abbiamo $f(x) = |x|$.\\
In questo caso, nel punto $x=0$, derivata destra e sinistra differiscono, perciò la derivata non esiste.
\subsection{Derivate da ricordare}
\begin{center}
 \begin{tabular}{|c|c|c|}
 \hline
 Funzioni & Derivate:\\
 \hline \hline
 $arcSin(x)$ & $\frac{1}{\sqrt{1-x^{2}}}$\\
 \hline
 $arcCos(x)$ & $-\frac{1}{\sqrt{1-x^{2}}}$\\
 \hline
 $arcTan(x)$ & $\frac{1}{1+x^{2}}$\\
 \hline
 \end{tabular}
\end{center}
\subsection{Trasformazioni trigonometriche da ricordare}
\begin{itemize}
    \item $sin(2x) = 2*sin(x)*cos(x)$
    \item $cos(2x) = cos^{2}x - sin^{2}x$
    \item $tan(x) = \frac{sin(x)}{cos(x)}$
    \item $sin^{2}x + cos^{2}x = 1$
\end{itemize}
\section{Teorema dell'esisistenza degli zeri e funzioni continue}
\subsection{Funzioni continue}
\textbf{Definizione di continuità}:\\
$\forall \epsilon > 0$ e $\exists \delta > 0$ e $\forall x \in Dom(f)$
then $$|x - x_{0}| < \epsilon \rightarrow |f(x) - f(x_{0}| < \delta$$.\\
Quindi $f(x) = \frac{1}{x}$, è una funzione continua, perchè $x_{0} \notin Dom(f)$\\
\textbf{Proprietà delle funzioni continue:} Se f è una funzione continua, allora: $$lim_{x \rightarrow 0}(f(g(x))) = f(lim_{x \rightarrow 0}(g(x)))$$
Le combinazioni di funzioni continue danno origine a funzioni continue. \\\textbf{Esempio:} polinomi, esponenzioli, e funzioni trigonometriche 
sono funzioni continue, quindi $\frac{sin(x^{3} + e^{x+2})}{x^{2}+1}$ è  continua.\\
\subsection{Teorema dell'esisistenza degli zeri}
Sia $[a,b]$ un intervallo chiuso, e $f: [a,b] \rightarrow R$, e sia che $f(a)$ e $f(b)$, abbiano segni opposti, 
allora $\exists x_{s} \in [a,b]$ tale che $f(x_{s}) = 0$\\
Per vedere se esiste uno zero, uso un algoritmo iterativo.\\
Ci sono due casi di partenza
\begin{enumerate}
    \item $f(a_{0}) < 0$ e $f(b_{0}) > 0$
    \item $f(a_{0}) > 0$ e $f(b_{0}) < 0$
\end{enumerate}
Analizziamo il primo caso:\\
Troviamo un punto $c_{0} = \frac{a_{0} + b_{0}}{2}$. A questo punto vediamo i casi:
\[
  f(c_{i})=\begin{cases}
               < 0 \rightarrow cerchiamo c_{i+1} a destra, I_{i+1} = [c_{i+1},b]\\
               = 0 \rightarrow abbiamo finito\\
               > 0 \rightarrow cerchiamo c_{i+1} a sinistra, I_{i+1} = [a,c_{i+1}]
            \end{cases}
\]
Continuo a cercare $c_{i}$, dimezzando ogni volta l'intervallo e cercando a destra o a sinistra in base all'algoritmo 
sovrastante, finche non trovo $f(c_{i}) = 0$; a quel punto ho trovato che esiste uno zero.\\

\textbf{Per dimostrare il teorema mi manca una cosa però}, l'algoritmo potrebbe continuare all'infinito, 
devo dimostrare quando e perchè non è cosi.\\
Vado a vedere la lunghezza dell'intervallo $I_{i}$.Guardo alcuni casi:
\begin{itemize}
    \item $I_{0} = [a_{0},b_{0}]$ e quindi la lunghezza di $I_{0}$ è uguale a $\frac{b_{0} - a_{0}}{1}$
    \item $I_{1} = [a_{1},b_{1}]$ e quindi la lunghezza di $I_{1}$ è uguale a $\frac{b_{0} - a_{0}}{2}$
    \item $I_{2} = [a_{2},b_{2}]$ e quindi la lunghezza di $I_{2}$ è uguale a $\frac{b_{0} - a_{0}}{4}$
    \item $I_{3}$ = ...
    \item $I_{n} = [a_{n},b_{n}]$ e quindi la lunghezza di $I_{n}$ è uguale a $\frac{b_{0} - a_{0}}{2^{n}}$
\end{itemize}
Visto che $I_{n} = \frac{b_{0} - a_{0}}{2^{n}} = b_{n} - a_{n}$ possiamo riscriverlo come $b_{n} - a_{n} = \frac{b_{0} - a_{0}}{2^{n}}$\\
Noi sappiamo che $a_{i} <= a_{i+1}$ sarà sempre vero, e allo stesso modo $b_{i} >= b_{i+1}$. Quindi possiamo concludere che 
la successione $a_{n}$ è \textbf{debolmente crescente} (vuol dire che è non decrescente, ovvero che non cresce per forza sempre, ma può anche rimanere uguale)
, mentre $b_{n}$ è \textbf{debolmente decrescente}.\\
Sappiamo che le successione monotone hanno sempre un limite, quindi
\begin{itemize}
    \item $lim_{x \rightarrow \infty}(a_{n}) = a_{s}$
    \item $lim_{x \rightarrow \infty}(b_{n}) = b_{s}$
\end{itemize}
$$a_{0} <= a_{s} <= b_{s} <= b_{0}$$
Riassumiamo le Proprietà dell'intervallo:
\begin{enumerate}
    \item Visto che $I_{n} = \frac{b_{0} - a_{0}}{2^{n}}$, sappiamo che $I_{n}$ tende a zero.
    \item $f(a_{i})$ e $f(b_{i})$ hanno segni opposti
    \item $a_{n}$ è debolmente crescente
    \item $b_{n}$ è debolmente decrescente
    \item Esiste $lim_{x \rightarrow \infty}(a_{n}) = a_{s}$ ed è finito
    \item Esiste $lim_{x \rightarrow \infty}(b_{n}) = b_{s}$ ed è finito
\end{enumerate}
Facciamo il limite di $b_{n} - a_{n} = \frac{b_{0} - a_{0}}{2^{n}}$, per trovare il nostro $x_{s}$:
\begin{align*}
    &lim_{n \rightarrow \infty}(b_{n} - a_{n}) = lim_{n \rightarrow \infty}(\frac{b_{0} - a_{0}}{2^{n}})\\
    &b_{s} - a_{s} = 0\\
    &a_{s} = b_{s}
\end{align*}
Poniamo $x_{s} = a_{s} = b_{s}$\\
\textbf{Inizio parte facoltativa (serve per dimostrare che x tocca tutti gli intervalli $I_{i})$}\\
Dobbiamo dimostrare che $a_{s} <= x_{s} <= b_{s}$.\\
Io so che $a_{0} <= a_{1} <= ... <= a_{n} < b_{n} <= b_{n-1} <= ... <= b_{0}$.\\

Preso $m < n$, allora $a_{0} <= a_{1} <= ... <= a_{m} <= ... <= a_{n} < b_{n} <= b_{n-1} <= ... <= b_{m} <= ... <= b_{0}$, 
e quindi $$a_{m} < b_{n}$$
Preso $m > n$, allora $a_{0} <= a_{1} <= ... <= a_{n} <= ... <= a_{m} < b_{m} <= b_{m-1} <= ... <= b_{n} <= ... <= b_{0}$, 
e quindi $$a_{n} < b_{m}$$
Considero $a_{n} < b_{m}$ (o anche l'altra, a scelta). Ora posso dire che:
\begin{align*}
    &a_{n} < b_{m}\\
    &lim_{n \rightarrow \infty}(a_{n}) <= b_{m}\\
    &x_{s} <= b_{m}
\end{align*}
\begin{center}
    e
\end{center}
\begin{align*}
    &a_{n} < b_{m}\\
    &a_{n} <= lim_{m \rightarrow \infty}(b_{m})\\
    &a_{n} <= x_{s}
\end{align*}
\textbf{Fine parte facoltativa}\\
Ora che abbiamo dimostrato la formula possiamo usarla:
\begin{align*}
    f(a_{n}) < 0\\
    lim_{n \rightarrow \infty}(f(a_{n})) <= 0\\
    f(x_{s}) <= 0
\end{align*}
\begin{center}
    e
\end{center}
\begin{align*}
    f(b_{n}) > 0\\
    lim_{n \rightarrow \infty}(f(b_{n})) >= 0\\
    f(x_{s}) >= 0
\end{align*}
Quindi, grazie a $f(x_{s}) <= 0$ e $f(x_{s}) >= 0$, posso dire che 
$$f(x_{s}) = 0$$
\textbf{Quindi esiste uno zero della funzione}.
\begin{center}
   \textbf{FINE DIMOSTRAZIONE}
\end{center}
\section{Studio di funzione}
\subsection{Asintoti verticali, orizzontali e obliqui}
Quando dobbiamo cercare degli asintoti in una funzione, dobbiamo prima di tutto verificare se, e quali sono 
i buchi della funzione, ovvero quei punti dove la funzione non è definita, oltre che a $+\infty$ e $-\infty$, e bisogna cercare gli asintoti in quei punti.\\
Un \textbf{asintoto verticale} lo si trova quando $lim_{x \rightarrow x_{0}^{+,-}} = +- \infty$.\\
Per esempio, se $lim_{x \rightarrow 0^{+}} = \infty$ o $lim_{x \rightarrow 0^{+}} = -\infty$ o $lim_{x \rightarrow 0^{-}} = \infty$ o $lim_{x \rightarrow 0^{-}} = -\infty$, 
allora abbiamo trovato un asintoto verticale.\\
Un \textbf{asintoto orizzontale} si trova quando $lim_{x \rightarrow \infty} = l$.\\
Per esempio, se $lim_{x \rightarrow +\infty} = l$ oppure $lim_{x \rightarrow -\infty} = l$, allora abbiamo un asintoto orizzontale.\\
Per trovare i punti di intersezione tra l'asintoto orizzontale e la funzione posso mettere a sistema la funzione e l'equazione dell'asintoto:\\
$$
\begin{cases}
    y = f(x)\\
    y = l
\end{cases}
$$
Un \textbf{asintoto obliquio} è tale se, data la retta $y = mx + q$, allora vale il limite $lim_{x \rightarrow \infty}(f(x) - mx - q)$.\\
Quando bisogna vedere se c'è un asintoto obliquo bisogna procedere in questo modo:\\
\begin{enumerate}
    \item dobbiamo calcolare $y = mx + q$, quindi calcoliamo $m = lim_{x \rightarrow \infty}(\frac{f(x)}{x})$; se $m \in R$, allora continuo (calcolo q), se $m = 0$ ho un asintoto orizzontale (calcolo q), altrimenti non ho un asintoto obliquo.
    \item $q = lim_{x \rightarrow \infty}(f(x) - mx)$. Se $q \in R$ allora ho un asintoto obliquo.
\end{enumerate}
Esempio:$f(x) = \frac{x^{2}}{x-1}$, trova se esiste $y = mx + q$
\begin{align*}
    &m = lim_{x \rightarrow \infty}(f(x)/x)\\
    &m = lim_{x \rightarrow \infty}(\frac{x^{2}}{x-1}/x)\\
    &m = lim_{x \rightarrow \infty}(\frac{1}{(1-\frac{1}{x})}) = 1\\
    &q = lim_{x \rightarrow \infty}(f(x) - mx)\\
    &q = lim_{x \rightarrow \infty}(\frac{x^{2}}{x-1} - x)\\
    &q = lim_{x \rightarrow \infty}(\frac{x}{x-1})\\
    &q = lim_{x \rightarrow \infty}(\frac{1}{1 - \frac{1}{x}}) = 1\\
    &y = mx + q = x + 1
\end{align*}
\subsection{Come affrontare una funzione con logaritmi o arctan e gli zeri della funzione}
Affrontare questo tipo di funzioni non è poi tanto diverso dall'affrontare quelle polinomiali, ma con la differenza che bisogna stare più attenti sia 
nella parte di studio della funzione di partenza e nel determinare il dominio, che nella successiva parte delle derivate.\\
In questo tipo di esercizi, può essere chiesto di trovare gli zeri della funzione. Torna utile quindi il teorema di esistenza degli zeri.\\
Come lo applico? Una volta che ho fatto la derivata prima, e lo studio della crescenza/descrescenza, posso controllare ogni intervallo uno ad uno, 
e vedere se esiste uno zero. Ricorda che la funzione $f:[a,b] \rightarrow R$, deve essere continua nell'intervallo $[a,b]$. Per controllare posso usare 
le conoscenze che ho già acquisito nei passi precedenti, infatti andrò a vedere cosa vale la mia funzione nel punto a e cosa vale nel punto b, e da questo posso concludere 
se la funzione è passata per lo zero.\\
Per esempio se la mia tabella di crescenza/descrescenza è:\\
\begin{tabular}{|c|c|c|}
    \hline
    $0$ & $\sqrt{\frac{1}{2}}$ & $1$:\\
    \hline \hline
    $-$ & $-$ & $+$\\
    \hline
    $+$ & $+$ & $+$\\
    \hline\hline
    $-$ & $-$ & $+$\\
    \hline
\end{tabular}\\
e conosco che in 1 c'è un asintoto verticale, allora potrò dire che nell'intervallo $[\sqrt{\frac{1}{2}},1]$ c'è uno zero:
\begin{itemize}
    \item $[-\infty,0]$ no
    \item $[0,\sqrt{\frac{1}{2}}]$ no
    \item $[\sqrt{\frac{1}{2}},1]$ si
    \item $[1,\infty]$ no
\end{itemize}
\section{Primitive e integrali}
\subsection{Primitive e tabella}
L'operazione di antiderivazione è l'opposto della derivazione, $$\int(\frac{d}{dx}f(x)) = f(x)$$ Vediamo un paio di trasformazioni da $f(x)$ a $\int f(x)$:
\begin{center}
    \begin{tabular}{|c|c|}
        \hline
        $f(x)$ & $\int f(x)$\\
        \hline\hline
        $k$(costante) & $kx$\\
        \hline
        $x^{a}$ & $\frac{x^{a+1}}{a+1}$\\
        \hline
        $\frac{1}{x}$ & $ln(|x|)$\\
        \hline
        $sen(x)$ & $-cos(x)$\\
        \hline
        $cos(x)$ & $sin(x)$\\
        \hline
        $a^{x}$ & $\frac{a^{x}}{log(a)}$\\
        \hline
        $\frac{1}{1+x^{2}}$ & $arctg(x)$\\
        \hline
    \end{tabular}
\end{center}
Vediamo le proprietà delle operazioni tra integrali e le funzioni composte
\begin{center}
    \begin{itemize}
        \item $\int(f(x) \pm g(x))dx = \int(f(x))dx \pm \int(g(x))dx$
        \item $\int(k*f(x))dx = k*\int(f(x))dx$
    \end{itemize}
\end{center}
Una regola fondamentale è quella che riguarda le funzioni composte:
$$\int f^{'}(g(x)) * g^{'}(x) = f(g(x))$$
Esempio:
\begin{align*}
    \int(3*x^{2} * sin(x^{3}))\\
    g^{'}(x) * f^{'}(g(x))\\
    -cos(x^{3}) + c
\end{align*}
Da questa regola ne derivano varie:\\
\begin{tabular}{|c|c|}
    \hline
    Primitiva elementare & Primitiva composta\\
    \hline\hline
    $f(x) = \frac{1}{x}$ e $\int f(x)dx = ln(|x|) + c$ & $\int(\frac{g^{'}(x)}{g(x)}) = ln|g(x)| + c$\\
    \hline
    $f(x) = x^{n}$ e $\int f(x)dx = \frac{x^{n+1}}{n+1} + c$ & $\int(g^{'}(x)*g(x)^{n}) = \frac{g(x)^{n+1}}{n+1}  + c$\\
    \hline
    \dots & \dots\\
    \hline
\end{tabular}
\subsection{Risoluzione delle primitive}
Ci sono 4 tipi di primitive delle verifiche di gorni: 1 composta da polinomi, 2 composte da log(x), cos(x), sin(x), tan(x), arctan(x), ecc., e 1 per parti.\\
Vediamo come risolvere queste tipologie separatamente.\\
\textbf{Primitive composte da polinomi}\\
Quando dobbiamo affrontare questo tipo di primitive, non possiamo partire in quarta e fare le primitive; prima bisogna fare la \textbf{Divisione del polinomio}. 
Ora vedremo questo primo procedimento:\\
\begin{enumerate}
    \item Denominiamo $P(x)$ il polinomio numeratore e $D(x)$ il polinomio denominatore;
    \item Scriviamo $P(x) | D(x)$;
    \item Calcoliamo $gMax = \frac{termine di grado max di P(x)}{termine di grado massimo di D(x)}$;
    \item Scriviamo
    \begin{tabular}{c|c}
        $P(x)$ & $D(x)$\\
        \hline
               & $gMax$
    \end{tabular}
    \item Moltiplichiamo gMax per ogni termine di P(x), e scriviamo i risultati, cambiati di segno, nella riga sotto P(x):\\
    \begin{tabular}{c|c}
        $P(x)$ & $D(x)$\\
        \hline
        $D(x)*gMax*(-1)$ & $gMax$
    \end{tabular}
    \item Sommiamo la riga $P(x)$ con la riga $P(x)*gMax*(-1)$ (somma termine per termine), e riportiamo il risultato nella riga sottostante:\\
    \begin{tabular}{c|c}
        $P(x)$ & $D(x)$\\
        \hline
        $D(x)*gMax*(-1)$ & $gMax$\\
        \hline
        $P(x) + D(x)*gMax*(-1)$
    \end{tabular}
    \item Continuo finchè non arrivo ad avere un grado zero in $Q(x)$:\\
    \begin{tabular}{c|c}
        $P1(x)$ & $D(x)$\\
        \hline
        $D(x)1*gMax1*(-1)$ & $Q(x) = gMax1$, $gMax2, \dots, gMaxN$\\
        \hline
        $P1(x) + D(x)1*gMax1*(-1)$\\
        \hline
        $D(x)*gMax2*(-1)$\\
        \hline
        $P2(x) + D2(x)*gMax2*(-1)$\\
        \hline
        $\dots$\\
        \hline
        $PN(x) = \dots$
    \end{tabular}
    \item Ora posso scrivere il risultato, che sarà $Q(x) + \frac{PN(x)}{D(x)}$
\end{enumerate}
Facciamo un esempio: $f(x) = \frac{3x^3-2x^2+x}{6x^2+5x+1}$\\
$P(x) = 3x^3-2x^2+x$ e $D(x) = 6x^2+5x+1$.\\
\begin{enumerate}
    \item \begin{tabular}{c|c}
        $3x^3-2x^2+x$ & $6x^2+5x+1$\\
        \hline
    \end{tabular}\\
    \item \begin{tabular}{c|c}
        $3x^3-2x^2+x$ & $6x^2+5x+1$\\
        \hline
                    & $\frac{x}{2}$\\
        \hline
    \end{tabular}\\
    \item \begin{tabular}{c|c}
        $3x^3-2x^2+x$ & $6x^2+5x+1$\\
        \hline
        $-3x^3 -\frac{5x^{2}}{2} - \frac{x}{2} + 0$ & $\frac{x}{2}$\\
        \hline
    \end{tabular}\\
    \item \begin{tabular}{c|c}
        $3x^3-2x^2+x$ & $6x^2+5x+1$\\
        \hline
        $-3x^3 -\frac{5x^{2}}{2} - \frac{x}{2} + 0$ & $\frac{x}{2}$\\
        \hline
        $0x^3 -\frac{9x^{2}}{2} + \frac{x}{2} + 0$ & \\
        \hline
    \end{tabular}\\
    \item \begin{tabular}{c|c}
        $3x^3-2x^2+x$ & $6x^2+5x+1$\\
        \hline
        $-3x^3 -\frac{5x^{2}}{2} - \frac{x}{2} + 0$ & $\frac{x}{2}$, $-\frac{3}{4}$\\
        \hline
        $0x^3 -\frac{9x^{2}}{2} + \frac{x}{2} + 0$ & \\
        \hline
    \end{tabular}\\
    \item \begin{tabular}{c|c}
        $3x^3-2x^2+x$ & $6x^2+5x+1$\\
        \hline
        $-3x^3 -\frac{5x^{2}}{2} - \frac{x}{2} + 0$ & $\frac{x}{2}$, $-\frac{3}{4}$\\
        \hline
        $0x^3 -\frac{9x^{2}}{2} + \frac{x}{2} + 0$ & \\
        \hline
        $0x^3 -\frac{18x^{2}}{4} + \frac{15x}{4} + 0$ & \\
    \end{tabular}\\
    \item \begin{tabular}{c|c}
        $3x^3-2x^2+x$ & $6x^2+5x+1$\\
        \hline
        $-3x^3 -\frac{5x^{2}}{2} - \frac{x}{2} + 0$ & $\frac{x}{2}$, $-\frac{3}{4}$\\
        \hline
        $0x^3 -\frac{9x^{2}}{2} + \frac{x}{2} + 0$ & \\
        \hline
        $0x^3 -\frac{18x^{2}}{4} + \frac{15x}{4} + 0$ & \\
        \hline
        $0x^3 + 0x^{2} + \frac{17x}{4} + \frac{3}{4}$ & \\
    \end{tabular}\\
    \item $PN(x) = \frac{17x}{4} + \frac{3}{4}$ e $Q(x) = \frac{x}{2} - \frac{3}{4}$.\\
    Quindi possiamo riscrivere
    \begin{align*}
        &f(x) = \frac{3x^3-2x^2+x}{6x^2+5x+1}\\
        &f(x) = Q(x) + \frac{PN(x)}{D(x)}\\
        &f(x) = \frac{x}{2} - \frac{3}{4} + \frac{\frac{17x}{4} + \frac{3}{4}}{6x^2+5x+1}
    \end{align*} 
\end{enumerate}
Ora abbiamo terminato con la prima semplificazione, se guardiamo il nostro esempio siamo nella situazione: $$f(x) = \frac{x}{2} - \frac{3}{4} + \frac{\frac{17x}{4} + \frac{3}{4}}{6x^2+5x+1}$$
La seconda parte è ancora troppo complicata per noi però. Ora il nostro obiettivo è semplificare $\frac{\frac{17x}{4} + \frac{3}{4}}{6x^2+5x+1}$.\\
Per farlo innanzitutto trasformiamo $6x^{2}+5x+1$ in una moltiplicazione di polinomi. Usando la formula per i polinomi $\frac{-b \pm \sqrt{b^{2}-4ac}}{2a}$, oppure ruffini scomponiamola.\\
Qui salto i passaggi: $6x^{2}+5x+1 = (2x+1)(3x+1)$. Quindi dobbiamo semplificare $\frac{1}{4} * \frac{17x+3}{(2x+1)(3x+1)}$.\\
Per farlo, suddivido la formula in questo modo, e risolvo il sistema:\\
\begin{enumerate}
    \item $\frac{A}{(2x+1)} + \frac{B}{(3x+1)}$\\
    \item $\frac{A(3x+1)+B(2x+1)}{(2x+1)(3x+1)}$\\
    \item $\frac{(3A+2B)x + (A+B)}{(2x+1)(3x+1)}$\\
    \item Risolvo il sistema:\\
    \begin{align*}
        \begin{cases}
            3A+2B = 17\\
            A+B = 3
        \end{cases}
        \begin{cases}
            A = 11\\
            B = -8
        \end{cases}
    \end{align*}
    \item Riscrivo l'equazione:\\
    $f(x) = \frac{x}{2} - \frac{3}{4} + \frac{\frac{17x}{4} + \frac{3}{4}}{6x^2+5x+1}$ = $\frac{x}{2} - \frac{3}{4} + \frac{1}{4} * \left(\frac{11}{2x+1} - \frac{8}{3x+1}\right)$\\
\end{enumerate}
\subsection{Integrali per parti}
Per risolvere un integrale per parti, uso questa formula $$\int(f(x)*g^{'}(x)) = f(x)*g(x) - \int(f^{'}(x)*g(x)) $$
Gli integrali per parti possono essere particolarmente ostici, e richiedere trucchi aggiuntivi per essere risolti.\\
Nei compitini gli integrali per parti saranno ciclici. Un integrale ciclico è un integrale $\int(f(x)*g^{'}(x))$, che dopo averlo integrato per parti, 
fa arrivare ad una situazione $\int(f(x)*g^{'}(x)) = f(x)*g(x) + h(x)*k(x) - \int(f(x)*g^{'}(x))$.\\
In questo caso abbiamo un integrale ciclico, perchè abbiamo riottenuto l'integrale originale, e per risolvere la situazione portiamo l'integrale che abbiamo ottenuto al primo membro e risolviamo. Ora vediamo un esempio:\\
\begin{align*}
    &\int(f(x)*g^{'}(x)) = f(x)*g(x) - \int(f^{'}(x)*g(x))\\
    &f(x)*g(x) + h(x)*k(x) - \int(h^{'}(x)*k(x)) ; (h^{'}(x) = f^{'}(x) , k(x) = g(x))\\
    &\int(f(x)*g^{'}(x)) = f(x)*g(x) + h(x)*k(x) - \int(f(x)*g^{'}(x))\\
    &2*\int(f(x)*g^{'}(x)) = f(x)*g(x) + h(x)*k(x)\\
    &\int(f(x)*g^{'}(x)) = \frac{f(x)*g(x) + h(x)*k(x)}{2}
\end{align*}
\subsection{Integrali per sostituzione}
Per risolvere gli integrali per sostituzione usiamo la formula $$\int(f(g(x))*g^{'}(x)) = \int_{g(a)}^{g(b)}(f(y)dy)$$
dove il punto del metodo per sotituzione è sostituire $x$ con $y$, eliminando i termini scomodi dalla formula per poter integrarla.\\
Applicazione del metodo per sostituzione:\\
\begin{enumerate}
    \item Sostituzione della $x$: $x = g(y)$
    \item Sostituzione della $dx$: $dx = \frac{d}{dy}(g(y))dy$
    \item Sostituzione degli intervalli dell'integrale (se presenti): $\int_{a}^{b}(f(x)dx)$: $\int_{g(a)}^{g(b)}(f(y)dy)$
\end{enumerate}
\section{Teorema di Fermat - Derivata di Massimi e Minimi}
\subsection{Punti interni di una funzione}
Prima diamo la definizione di \textbf{punto interno}: un punto interno di una funzione è 
un punto $x_{0} \in dom(f), \exists \delta > 0 : [x_{0} - \delta, x_{0} + \delta]$.\\
Quindi un punto interno è un punto all'interno del dominio di una funzione, tale che se andiamo leggermente a dx o a sx nel dominio, rimaniamo ancora 
nel dominio. Per esempio se il mio dominio è $dom(f) = [1,3]$, il valore $1.1$ è nel dominio perchè posso avere i valori $1.0$ e $1.2$, invece il valore 
$1$, non è nel dominio poichè anche se posso avere il valore $1.1$, non posso avere $0.9$, o valori minori di $1$, poichè sarebbero fuori dal dominio.\\
In breve possiamo riassumere i punti interni di una funzione come tutti quei punto che stanno dentro ai vari intervalli che compongono il dominio, e che 
non sono gli estremi di nessun intervallo.\\
\subsection{Teorema di Fermat}
Dato un punto interno $x_{0}$ della funzione f, dove $x_{0}$ è un max/min locale, e $f^{'}(x_{0})$ esiste. Allora $f^{'}(x_{0}) = 0$.\\
\textbf{DIMOSTRAZIONE}\\
Faccio il caso in cui $x_{0}$ è un punto di max. Lo considero max globale quindi $\forall x, f(x) \le f(x_{0})$. Devo dimostrare $f^{'}(x_{0}) = 0$.\\
Faccio la derivata come limite del rapporto incrementale: $lim_{x \rightarrow x_{0}}\frac{f(x)-f(x_{0})}{x-x_{0}}$, e provo a fare i limiti dx e sx per vedere come si comporta la derivata.\\
Per dimostrare $lim_{x \rightarrow x_{0^{+}}}\frac{f(x)-f(x_{0^{+}})}{x-x_{0^{+}}}$, suppongo che $x > 0$, allora studio il segno di $g(x) = \frac{f(x)-f(x_{0})}{x-x_{0}}; x > x_{0}$ e ottengo che il numeratore è 
maggiore di zero, mentre il denominatore è maggiore di zero, quindi la funzione è minore di zero: $g(x) <= 0$. Quindi anche il $lim_{x \rightarrow x_{0^{+}}} g(x_{0^{+}}) <= 0$\\
Facendo la stessa cosa per il limite sx ottengo:\\
\begin{itemize}
    \item $lim_{x \rightarrow x_{0^{-}}} = \frac{f(x) - f(x_{0^{-}})}{x - x_{0^{-}}}$
    \item $h(x) = \frac{f(x)-f(x_{0})}{x-x_{0}}; x < x_{0}$, quindi sia numeratore che denominatore sono negativi, e la funzione è positiva $h(x) >= 0$
    \item Visto che $h(x) \ge 0$, allora anche $lim_{x \rightarrow x_{0^{-}}}g(x_{0^{-}}) \ge 0$ = $f^{'}(x_{0^{-}}) \ge 0$.
\end{itemize}
Quindi visto che $f^{'}(x_{0^{-}}) \ge 0$ e $f^{'}(x_{0^{+}}) \le 0$, allora abbiamo dimostrato che $f^{'}(x_{0}) = 0$.
\begin{center}
    \textbf{FINE DIMOSTRAZIONE}
\end{center}
\section{Teoremi del valore medio: Rolle, Cauchy e Lagrange}
\subsection{Teorema di Rolle}
Data una funzione f definita su di un intervallo $[a,b]$, e con le seguenti condizioni:\\
\begin{itemize}
    \item la funzione f è continua in $[a,b]$,
    \item l'intervallo aperto $(a,b)$ è derivabile,
    \item $f(a) = f(b)$.
\end{itemize}
Allora esiste almeno un punto $c \in (a,b)$, tale che $f^{'}(c) = 0$.\\
Per dimostrare questo teorema ci serviremo del teorema dei valori estremi (teorema di weierstrass), e del teorema di Fermat.\\
\begin{itemize}
    \item Teorema dei valori estremi: se ho f continua, definita in un intervallo $[a,b]$, allora $\exists c, \exists d$, tale che 
    $f(c) \ge f(x) \ge f(d)$ con $c,d \in [a,b]$. (in breve ci sono ed esistono i massimi e minimi in una funzione continua in $[a,b]$).
    \item Teorema di Fermat: se ho f continua in $[a,b]$, preso un punto interno $x_{0}$ max o min locale/globale, allora $f^{'}(x_{0}) = 0$.
\end{itemize}
Dobbiamo verificare tre casi:\\
\begin{enumerate}
    \item se $f(x) = f(c) \forall x$, allora il grafico è una retta orizzontale, e ogni punto $f^{'}(c) = 0$.
    \item se $f(x) > f(a)$ per qualche $x \in (a,b)$, e f è una funzione continua definita in un intervallo $[a,b]$, allora esiste un valore c, compreso tra $[a,b]$, ed
    è un massimo. Ma visto che $f(x) > f(a)$ per qualche $x \in (a,b)$, allora $c \in (a,b)$ ed è un massimo. Se c è un massimo, allora per il teorema di Fermat, $f^{'}(c) = 0$.
    \item se $f(x) < f(b)$ per qualche $x \in (a,b)$, e f è continua in $[a,b]$, allora esiste un valore $x = c$, tale che c è un minimo. Visto che 
    $f(x) < f(b)$ per qualche $x \in (a,b)$, allora $c \in (a,b)$. Quindi per il teorema di Fermat, $f^{'}(c) = 0$, perchè c è un minimo.
\end{enumerate}
\begin{center}
    \textbf{FINE DIMOSTRAZIONE}
\end{center}
\subsection{Teorema di Cauchy}
Date due funzioni f e g, continue in un intervallo $[a,b]$, e derivabili in $(a,b)$, allora esiste almeno un punto interno $x_{0}$, tale che 
$$[g(b)-g(a)]*f^{'}(x_{0}) = [f(b)-f(a)]*g^{'}(x_{0})$$
Dimostrazione:\\
Per prima cosa scriviamo $h(x) = [f(b)-f(a)]*g(x) - [g(b)-g(a)]*f(x)$, e valutiamo $h(a)$ e $h(b)$.\\
\begin{align*}
    &h(a) = g(a)*f(b) - g(a)*f(a) - f(a)*g(b) + f(a)*g(a)\\
    &h(a) = g(a)*f(b) - f(a)*g(b)\\
    &h(b) = g(b)*f(b) - g(b)*f(a) - f(b)*g(b) + f(b)*g(a)\\
    &h(b) = g(b)*f(a) - f(b)*g(a)\\
    &h(a) = h(b)
\end{align*}
Ora noi possiamo applicare il teorema di rolle, perchè rispettiamo le tre condizioni: continuità di f e g su $[a,b]$, derivabilità su $(a,b)$, e $h(a) = h(b)$.\\
Quindi se facciamo la derivata di $h(x)$ otteniamo $$h^{'}(x) = [f(b)-f(a)]*g^{'}(x) - [g(b)-g(a)]*f^{'}(x)$$ e quindi se valutiamo la funzione h in $x_{0}$, possiamo dire:\\
\begin{align*}
    &h^{'}(x_{0}) = [f(b)-f(a)]*g^{'}(x_{0}) - [g(b)-g(a)]*f^{'}(x_{0})\\
    &0 = [f(b)-f(a)]*g^{'}(x_{0}) - [g(b)-g(a)]*f^{'}(x_{0})\\
    &[g(b)-g(a)]*f^{'}(x_{0}) = [f(b)-f(a)]*g^{'}(x_{0})
\end{align*}
Abbiamo ottenuto l'ipotesi iniziale, e dimostrato il teorema.
\begin{center}
    \textbf{FINE DIMOSTRAZIONE}
\end{center}
\subsection{Teorema di Lagrange}
Date due funzioni f continua nell'intervallo $[a,b]$, e derivabili in $(a,b)$, allora esiste un valore $x_{0}$ interno, tale che $$f(b)-f(a) = f^{'}(x_{0})*[b-a]$$
Per la dimostrazione usiamo la funzione ausiliaria $h(x) = f(x) - f(a)*-\frac{f(b)-f(a)}{b-a}*(x-a)$. Sappiamo che:\\
\begin{itemize}
    \item $f(x)$ è continua e derivabile
    \item $f(a)$ è continua e derivabile perchè è una costante
    \item $\frac{f(b)-f(a)}{b-a}$ è continua e derivabile perchè è una costante
    \item $(x-a)$ è continua e derivabile perchè è un polinomio
\end{itemize}
Per dimostrare $h^{'}(x_{0}) = 0$, dimostriamo $h(a) = h(b)$:\\
\begin{align*}
    &h(a) = f(a) - f(a)*-\frac{f(b)-f(a)}{b-a}*(a-a) = 0\\
    &h(b) = f(b) - f(a)*-\frac{f(b)-f(a)}{b-a}*(b-a) = 0\\
    &h(a) = h(b) = 0
\end{align*}
Quindi per il teorema di Rolle, visto che h(x) è continua in $[a,b]$, derivabile in $(a,b)$ e $h(a) = h(b)$, allora esiste $x_{0}$ tale che $h^{'}(x_[0]) = 0$.\\
Quindi se deriviamo $h^{'}(x)$, otteniamo $h^{'}(x) = f^{'}(x) - f^{'}(a)*-\frac{d}{dx}(\frac{f(b)-f(a)}{b-a}*(x-a))$, e sostituendo $x = x_{0}$, otteniamo $0 = f^{'}(x) -\frac{f(b)-f(a)}{b-a}$, 
e quindi abbiamo dimostrato la tesi perchè $f^{'}(x) = \frac{f(b)-f(a)}{b-a}$.\\
\begin{center}
    \textbf{FINE DIMOSTRAZIONE}
\end{center}
\end{document}
