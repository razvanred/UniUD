\documentclass[11pt]{article}

\begin{document}
\title{Deduzione Naturale}
\author{Calabrigo Massimo}
\date{\today}
\maketitle

\tableofcontents

\section{Deduzione naturale}
\subsection{Cos'è la deduzione naturale?}
La deduzione naturale è una tecnica utilizzata per trovare la conseguenza logica tra due formule A e B, 
e nella deduzione si scrive come $A \triangleright B$, oppure A deduce B.\\
Una tecnica utile per risolvere esercizi della deduzione naturale, in una formula $A \triangleright B$
 è quella di procedere dalla tesi B alle ipotesi A, in modo da capire di cosa si ha bisogno per arrivare alla fine, e "ricorsivamente" 
 costruire l'albero, ovviamente bisogna tener conto anche delle ipotesi, e quindi il modo 
 giusto di risolvere gli esercizi è:
 \begin{enumerate}
     \item Guardare le ipotesi iniziali e tenerle a mente
     \item Guardare la tesi e vedere come si può scomporre al livello sovrastante
     \item Continuare a scomporre le ipotesi finche si riesce, e quando ci si sente sicuri provare a 
     sviluppare le ipotesi per ottenere le parti scomposte della tesi
     \item Aggiungere, se servono, ipotesi che verranno successivamente scaricate, in modo da ottenere le parti scomposte della tesi, 
     e quando si deve decidere che ipotesi aggiungere, chiedersi sempre se poi queste ipotesi potranno essere scaricate.
     \item Fare in ogni momento un \textbf{elenco delle ipotesi da scaricare}. Se si riesce arrivare alla fine dell'esercizio con "ipotesi infinite",
      e poi fare un elenco di tutte le ipotesi e vedere se l'esercizio può funzionare, o ha bisogno di modifiche.
     \item Finito
 \end{enumerate}
\subsection{Formule della deduzione naturale}
\textbf{Formule normali}
\begin{enumerate}
    \item $ F,G| F and G$; (and i)
    \item $F and G|F$; (and e.1)
    \item $F and G|G$; (and e.2)
    \item $F|F or G$; (or i.1)
    \item $G|F or G$; (or i.2)
    \item $T,[F] \triangleright G|F \rightarrow G$; ($\rightarrow$ i)
    \item $F,F \rightarrow G| G$; ($\rightarrow$ e)
    \item $F, \neg F| \bot$ ($\neg$ e)
    \item $[F],[G] \triangleright (F and G, H, H)| H$; (or e)
    \item $[F] \triangleright \bot | \neg F$; ($\neg$ i)
    \item $\neg \neg F| F$; ($\neg \neg$ e)
\end{enumerate}
\textbf{Formule speciali}
\begin{enumerate}
    \item $T \triangleright \bot| F$
    \item $[F or \neg F]$
    \item $T,[F] \triangleright \bot| \neg F$
    \item $F \rightarrow G, \neg G| \neg F$
\end{enumerate}
\subsection{Dettagli utili per la deduzione naturale, e cose poco intuitive}
\begin{enumerate}
    \item Quando sono in una situazione di ($i \Rightarrow$ ), 
    dove normalmente la regola sarebbe $$T,F \triangleright G|(F \rightarrow G)$$ 
    se ho già scaricato F da un altra parte allora posso fare direttamente 
    $$G|(F \rightarrow G)$$ senza rifare lo scaricamento (ma devo comunque segnare 
    affianco all linea tra $G| G \rightarrow F$, che sto facendo uno scaricamento
    mettendo il solito numeretto).
    \item Quando sono in una situazione di \textbf{ex-falso}, con una formula del tipo
    $$ T \triangleright \bot|F $$ significa che avendo un falso, da esso posso
    dedurre qualsiasi formula.
    \item Nella regola (or e), si puà usare come primo campo $F or \neg F$, che poi verrà automaticamente scaricato.
\end{enumerate}



\end{document}