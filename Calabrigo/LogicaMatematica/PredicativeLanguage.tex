\documentclass[11pt]{article}

\begin{document}

\title{Linguaggio Predicativo}
\author{Massimo calabrigo}
\date{\today}
\maketitle

\tableofcontents

\section{Variabile Libera}
Una variabile libera è una variabile che non è legata ai quantificatori ($\exists$ e $\forall$).\\
Quindi, se abbiamo una formula del tipo $\forall x(r(x, y))$, allora $y$ sarà una variabile libera, mentre $x$ sarà una variabile legata.
Possiamo pensare più facilmente ad $y$ come una variabile vera e propria, alla quale posso sostituire qualsiasi valore io voglia (presente nel dominio), mentre
per quanto riguarda la $x$, è una valore al quale dobbiamo sostituire, uno alla volta, tutti i valori del dominio.
\section{Sostituzione di termini e/o formule}
La scrittura $r(f(x)) \{x/t\} = r(f(t))$, significa che devi sostituire alla variabile x, il termine t.\\
La scrittura $r(f(x)) \{x/f(x)\} = r(f(f(x)))$, significa che devo sostituire alla variabile x, il termine f(x).\\
Per poter fare una sostituzione, devo essere sicuro che la variabile che sostituisco sia una variabile libera, 
inoltre devo accertarmi che, dopo averla sostituita, essa rimanga ancora libera.Esempio:
$$\forall x(r(x,y)) \{x/w\}$$
Nella formula sovrastante ci sono 2 variabili: $x$ e $y$. $x$ è legata (dal quantificatore $\forall x$) mentre $y$ è libera, e visto 
che io sto cercando di sostituire $w$ ad $x$, e che dopo aver sostituito $w$, quest'ultima sarà comunque una variabile libera, poichè 
non è legata a nessun quantificatore, allora posso effettuare la sostituzione.
\section{Interpretazione di una formula}
Cos'è un linguaggio?\\
Un linguaggio è composto da simboli di costante, di relazione e di funzione, oltre che a quantificatori e connettivi logici.\\
Un linguaggio può anche essere definito come insieme di formule.\\
Cos'è l'interpretazione di una formula?\\
L'interpretazione è il valore che assume quella formula, al variare dello stato (vedi dopo) a cui è associata. Si scrive $I, \sigma$ soddisfa $F$, 
dove F è la formula e $\sigma$ è lo stato.\\
Per trovare l'interpretazione di una formula, dobbiamo avere l'interpretazione di un linguaggio, per esempio se dovessimo avere un 
linguaggio $L_{0}$, con 1 simbolo di costante c, un simbolo di relazione binario r e un simbolo di funzione f, 
una interpretazione possibile potrebbe essere:
\begin{itemize}
    \item $ D = \{0,1,2\}$
    \item $c^{I} = 1$
    \item $f(0) = 1, f(1)=2,f(2)=3$
    \item $r^{I} = \{(0,0),(1,2),(2,2)\}$
\end{itemize}
\subsection{Stato di una formula}
Lo stato associato ad una formula, è una funzione che manda da una qualsiasi variabile, ad un elemento
 del dominio: $\sigma : Var \to D^{I}$.\\
Lo stato si scrive in coppia con l'interpretazione, e li si usano per vedere se una determinata formula F, possa far parte dell'insieme 
delle formule T del linguaggio $L_{0}$. Si scrive $I, \sigma$ soddisfa $F$. E si legge l'interpretazione con stato sigma, soddisfa F.\\
Uno stato possibile potrebbe essere:
\begin{itemize}
    \item $\sigma(x) = 0$
    \item $\sigma(y) = 1$
    \item $\sigma(w) = 2$ (con $w \neq x,y$)
\end{itemize}
Il sigma può essere usato su qualsiasi termine, ma avrà effetti diversi:
\begin{itemize}
    \item (termine) $\sigma(variabile) \to \sigma(x)$, per i $\sigma(Var)$ definiti nello stato (esempio sopra)
    \item (termine) $\sigma(costante) \to \sigma(c^{I})$
    \item (termine) $\sigma(simbolo di funzione) \to f(\sigma(t_1),\sigma(t_2), ... ,\sigma(t_n))$
    \item (formula) $\sigma(simbolo di relazione) \to$ risolvo i sigma, e vedo se $p(0,1,3)$ appartiene a $p^{I}\{(1,2,3),(3,6,8), ... ,(2,8,34)\}$. Se 
    appartiene allora $I, \sigma $ soddisfa $p(0,1,3)$, altrimenti no.
    \item (formula) Per tutti i tipi di formule composte da connettivi logici, tranne i quantificatori, il soddisfa funziona in modo analogo alla soffisfazione del connettivo scelto, 
    per esempio se ho $I, \sigma $ soddisfa F or G, deve valere $I, \sigma $ soddisfa F oppure $I, \sigma $ soddisfa G.
    \item (formula) Per le formule con un quantificatore devo fare:
    \begin{itemize}
        \item Quantificatore Esistenziale: $D^{I} = \{0,1,2\}$ ($\forall x (r(x,y)$)), devo controllare che $I, \sigma$ soddisfi 
        $r(x,y)$, per almeno un valore di x, quindi devo verificare:
        \begin{enumerate}
            \item $I, \sigma[x/0]$ soddisfa $r(x,y)$ oppure
            \item $I, \sigma[x/1]$ soddisfa $r(x,y)$ oppure
            \item $I, \sigma[x/2]$ soddisfa $r(x,y)$
        \end{enumerate}
        \item Quantificatore Per ogni: $D^{I} = \{0,1,2\}$ ($\forall x (r(x,y)$)), devo controllare che $I, \sigma$ soddisfi 
        $r(x,y)$, per tutti i valori di x, quindi devo verificare:
        \begin{enumerate}
            \item $I, \sigma[x/0]$ soddisfa $r(x,y)$ e
            \item $I, \sigma[x/1]$ soddisfa $r(x,y)$ e
            \item $I, \sigma[x/2]$ soddisfa $r(x,y)$
        \end{enumerate}
    \end{itemize}
\end{itemize}
\subsection{Metodologia da seguire per l'intepretazione di una formula}
Dati Una formula F, uno stato $\sigma$ e un'interpretazione I, bisogna trovare se $I, \sigma$ soddisfi F. Vedi i punti sovrastanti, 
c'è scritto cosa fare ad ogni passo!
\section{Equivalenza e Conseguenza Logica}
Due formule A e B sono logicamente equivalenti $A \equiv B$, se per ogni interpretazione I, 
e ogni stato $\sigma$: $I, \sigma$ soddisfa $A$ se e solo se $I, \sigma$ soddisfa $B$.\\
Date 2 formule A e B, B è conseguenza logica di A se per ogni interpretazione I, 
e ogni stato $\sigma$: $I, \sigma$ soddisfa $A$ si ha $I, \sigma$ soddisfa $B$
\end{document}