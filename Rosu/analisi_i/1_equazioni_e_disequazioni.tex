\documentclass[12pt, letterpaper]{article}
\usepackage[utf8]{inputenc}
\usepackage{amssymb, amsmath, amsfonts}
\usepackage{tcolorbox}
\usepackage{commath}

\newtheorem{definition}{Definizione}[section]
\newtheorem{principle}{Principio}

\title{Analisi Matematica 1}
\author{Razvan Rosu}
\date{\today}

\begin{document}

	\begin{titlepage}
		\maketitle
	\end{titlepage}

	\section{Equazioni e Disequazioni}
	
	\begin{definition}[Disequazione]
		Una disequazione è una disuguaglianza in cui compaiono espressioni letterali per le quali cerchiamo i valori di una o più lettere che rendono la disuguaglianza vera.
	\end{definition}
	
	Le lettere per i quali si cercano i valori sono le \textbf{incognite}. I valori delle incognite che rendono vera la disequazione sono le \textbf{soluzioni} della disequazione. \par
	Una disequazione è
	
	\begin{itemize}
		\item \textbf{numerica} se nell'equazione non compaiono altre lettere all'incognita
		\item \textbf{letterale} se contiene altre lettere, che possono essere chiamati anche \textit{parametri}.
		\item \textbf{intera} se l'incognita compare soltanto nei numeratori delle eventuali frazioni presenti nella disequazione
		\item \textbf{fratta} se l'incognita è contenuta nel denominatore di qualche frazione
	\end{itemize}

	\subsection{Gli intervalli}
	
	Spesso gli insiemi di soluzioni delle disequazioni sono particolari sottoinsiemi di $\mathbb{R}$ chiamati \textbf{intervalli}.
	
	\begin{definition}[Intervallo limitato]
		Dati due numeri reali $a$ e $b$, con $a < b$, si chiama intervallo limitato l'insieme dei numeri reali $x$ compresi tra $a$ e $b$. 
	\end{definition}

	\begin{definition}[Intervallo illimitato]
		Dato un numero reale $a$, si chiama intervallo illimitato l'insieme dei numeri reali $x$ che precedono $a$, oppure l'insieme dei numeri reali $x$ che seguono $a$.
	\end{definition}

	\subsection{Le disequazioni equivalenti}
	
	\begin{definition}[Disequazioni equivalenti]
		Due disequazioni si dicono equivalenti se hanno lo stesso insieme di soluzioni.
	\end{definition}

	Valgono i seguenti principi.
	
	\begin{principle}[Primo principio di equivalenza]
		Data una disequazione, si ottiene una disequazione ad essa equivalente aggiungendo ad entrambi i membri uno stesso numero o espressione.
	\end{principle}

	In questo senso, possiamo dire che \textbf{un termine può essere trasportato da un membro all'altro della disequazione cambiandogli il segno}.
	
	\begin{principle}[Secondo principio di equivalenza]
		
		Data una disequazione, si ottiene una disequazione ad essa equivalente:
		
		\begin{itemize}
			\item moltiplicando o dividendo entrambi i membri per uno stesso numero (o espressione) \textit{positivo}.
			\item moltiplicando o dividendo entrambi i membri per un numero (o espressione) \textit{negativo} e \textit{cambiando il verso} della disuguaglianza.
		\end{itemize}
	
	\end{principle}
	
	In particolare, \textbf{se si cambia il segno di tutti i termini di una disequazione e si inverte il verso della diseguaglianza, si ottiene una disequazione equivalente}.
	
	\section{Le disequazioni di primo grado}
	
	Le disequazioni intere di primo grado possono sempre essere scritte in una delle seguenti forme, dopo aver opportunamente applicato i principi di equivalenza:
	
	\begin{center}
		$ax > b$ ~ $ax >= b$ ~ $ax < b$ ~ con $a, b \in \mathbb{R}$
	\end{center}

	Risolvendo $ax > b$, otteniamo, a seconda dei valori di $a$:
	
	\begin{itemize}
		\item se $a > 0$, ~ $x > \tfrac{b}{a}$;
		\item se $a = 0$
		\begin{itemize}
			\item se $b > 0$, ~ $S = \O$
			\item se $b = 0$, ~ $S = \O$
			\item se $b < 0$, ~ $S = \mathbb{R}$
		\end{itemize}
		\item se $a < 0$, ~ $x < \tfrac{b}{a}$
	\end{itemize}

	Discutere le soluzioni di una disequazione letterale permette di ottenere le soluzioni di infinite disequazioni numeriche, quelle che si hanno sostituendo nell'equazione data valori particolari alla lettera (o alle lettere).
	
	\subsection{Lo studio del segno di un prodotto}
	
	Consideriamo una disequazione costituita da un prodotto di binomi di primo grado messo a confronto con il numero 0, per esempio
	
	\[
		(x - 1)(3x + 2)(x + 4) > 0.
	\]
	
	Per risolverla possiamo \textbf{studiare il segno} del prodotto al variare di $x$.
	
	\[x - 1 > 0 ~ \rightarrow ~ x > 1\]
	
	\[3x + 2 > 0 ~ \rightarrow ~ x > -\frac{2}{3}\]
	
	\[x + 4 > 0 ~ \rightarrow ~ x > -4\]
	
	La disequazione richiede che il prodotto sia positivo, quindi l'insieme delle soluzioni è:
	
	\[-4 < x < -\frac{2}{3} ~ \lor ~ x > 1\]
	
	\section{Le disequazioni di grado superiore al secondo e le disequazioni fratte}
	
	\subsection{Le disequazioni di grado superiore al secondo}
	Dato un polinomio $P(x)$ di grado maggiore di 2, le disequazioni del tipo $P(x)<0$ o $P(x)>0$ sono di grado superiore al secondo e possono essere risolte scomponendo in fattori di primo e secondo grado il polinomio $P(x)$ e studiando il segno del prodotto di polinomi che si ottiene.
	
	\subsection{Le disequazioni fratte}
	Una disequazione è \textbf{fratta} se contiene l'incognita al denominatore. Può essere sempre trasformata in una disequazione del tipo
	
	\[
		\frac{A(x)}{B(x)} > 0
	\]
	
	o in altre analoghe con i diversi segni della disuguaglianza.
	
	Per risolvere una disequazione fratta dobbiamo studiare il segno della frazione $\frac{A(x)}{B(x)}$, esaminando quelli di $A(x)$ e di $B(x)$. Dobbiamo imporre $B(x) != 0$ per la condizione di esistenza della frazione.
	
	\section{I sistemi di disequazioni}
	
	\begin{definition}[Sistema di disequazioni]
		Un sistema di disequazioni è un insieme di più disequazioni nella stessa incognita, per le quali cerchiamo le soluzioni comuni.
	\end{definition}
	
	Le \textbf{soluzioni} del sistema sono quei numeri reali che soddisfano \textit{contemporaneamente} tutte le disequazioni.
	
	\section{Le equazioni e le disequazioni con valore assoluto}
	
	Il valore assoluto di un numero è uguale al numero stesso se il numero è positivo o nullo, è l'opposto del numero se il numero è negativo. In generale:
	
	\begin{tcolorbox}
			\[
			\abs{x} = 
			\begin{cases}
				x, & \mbox{ se } x >= 0\\
				-x, & \mbox{ se } x < 0
			\end{cases}
			\]
	\end{tcolorbox}
	
	Alcune proprietà del valore assoluto:
	
	\begin{enumerate}
		\item $\abs{x} = \abs{-x} \qquad \forall x \in \mathbb{R}$
		\item $\abs{x * y}=\abs{x} * \abs{y} \qquad \forall x, y \in \mathbb{R}$
		\item $\abs{\frac{x}{y}} = \frac{\abs{x}}{\abs{y}} \qquad \forall x, y \in\mathbb{R}, y \neq 0$
		\item $\abs{x} = \abs{y} \quad \Leftrightarrow \quad x = \pm{y} \qquad \forall x, y \in \mathbb{R}$
		\item $\abs{x} <= \abs{y} \quad \Leftrightarrow \quad x^2 \leq y^2 \qquad \forall x, y \in \mathbb{R}$
		\item $\sqrt{x^2} = \abs{x} \qquad \forall x \in \mathbb{R}$
	\end{enumerate}
		
	
\end{document}